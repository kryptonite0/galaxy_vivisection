
\documentclass[preprint2]{emulateapj}

\bibliographystyle{apj}
\usepackage{natbib}
\usepackage{longtable}
\usepackage[]{graphicx}
\usepackage{amsmath}
\usepackage{tabularx}
\usepackage{bm}
\usepackage{color}
\usepackage{hyperref}


\newcommand{\vdag}{(v)^\dagger}
\newcommand{\myemail}{gsavorgn@astro.swin.edu.au}
\newcommand{\fitfigurewidth}{0.8\textwidth}


\shorttitle{Mbh-n paper}
\shortauthors{Savorgnan et al.}

\begin{document}


\title{Mbh-n paper}


\author{G. A. D. Savorgnan\altaffilmark{1} and A. W. Graham\altaffilmark{1}}
\affil{Centre for Astrophysics and Supercomputing, Swinburne University of Technology, Hawthorn, Victoria 3122, Australia.}
\email{gsavorgn@astro.swin.edu.au}


\begin{abstract}
blah blah
\end{abstract}


\keywords{keywords}


\section{Introduction}
The empirical \cite{sersic1963,sersic1968} $R^{1/n}$ model has been demonstrated to provide adequate description 
of the light distribution of the stellar spheroidal and disk components of galaxies (add REFS), 
yet its physical origin has remained unexplained for decades. 
The S\'ersic model parametrizes the intensity of light $I$ as a function of the projected galactic radius $R$ such that
\begin{equation}
I(R; I_{\rm e},R_{\rm e},n) = I_{\rm e} \exp \biggl \{ -b_{\rm n} \biggl [ \Bigl (\frac{R}{R_{\rm e}} \bigr )^{1/n} -1 \biggr ] \biggr \}, 
\end{equation}
where $I_{\rm e}$ indicates the intensity at the effective radius $R_{\rm e}$ that encloses half of the total light from the model, 
the S\'ersic index $n$ is the parameter that regulates the curvature of the radial light profile, 
and $b_{\rm n}$ is a constant defined in terms of the S\'ersic index $n$ (see \citealt{grahamdriver2005}, and references therein). 
A large S\'ersic index corresponds to a steep inner profile and a shallow outer profile, 
whereas a small S\'ersic index corresponds to a shallow inner profile and a steep outer profile. 
This means that, for a stellar spheroidal system whose light distribution is well approximated by the S\'ersic model, 
the larger the S\'ersic index is, the more centrally concentrated the stars are and the more extended the outer envelope is. \\
A compelling physical interpretation for the S\'ersic profile family was recently theorized by \cite{cen2014} 
and later confirmed by \cite{nipoti2015} by means of $N$-body simulations. 
\cite{cen2014} conjectured that, when structures form within a standard cold dark matter model seeded by random Gaussian fluctuations, 
any centrally concentrated stellar structure always possesses an extended stellar envelope, and vice versa. 
\cite{nipoti2015} quantitatively explored Cen's hypothesis and showed that 
systems originated from several mergers have a large S\'ersic index ($n \gtrsim 4$), 
whereas systems with a S\'ersic index as small as $n \simeq 2$ can be produced by coherent dissipationless collapse, 
and exponential profiles ($n=1$) can only be obtained through dissipative processes.
This scenario sets the theoretical framework for the well known correlation between the spheroid luminosity, $L_{\rm sph}$,  
and the spheroid S\'ersic index, $n_{\rm sph}$, (e.g.~\citealt{youngcurrie1994,jerjen2000,grahamguzman2003}), 
although the numerical results of \cite{nipoti2015} seem to lack of spheroidal systems with S\'ersic indices as large as $7 - 10$, 
which are commonly observed in the local Universe.  \\
Given the existence of the $L_{\rm sph} - n_{\rm sph}$ correlation 
and the relation between the central black hole mass, $M_{\rm BH}$, and the spheroid luminosity (e.g.~\citealt{magorrian1998}), 
an $M_{\rm BH} - n_{\rm sph}$ relation must exist. 
After \cite{graham2001} showed that the black hole mass is tightly linked to the stellar light concentration of spheroids 
(measured through a parameter different from, but closely related to the S\'ersic index), 
\cite{grahamdriver2007} presented for the first time the $M_{\rm BH} - n_{\rm sph}$ correlation 
using a sample of 27 elliptical and disk galaxies. 
\cite{grahamdriver2007} fit their data with a log-quadratic regression, 
finding that the $M_{\rm BH} - n_{\rm sph}$ log-relation is steeper for spheroids with small S\'ersic indices 
and shallower for spheroids with large S\'ersic indices, 
and measured a relatively small level of scatter\footnote{At the time, the $M_{\rm BH} - \sigma$ relation 
\citep{ferraresemerritt2000,gebhardt2000} was reported to have the same level of scatter as the $M_{\rm BH} - n_{\rm sph}$ 
relation ($\simeq 0.3~\rm dex$). }.
A few years later, \cite{sani2011}, \cite{vika2012} and \cite{beifiori2012} performed multi-component decompositions 
for samples of galaxies similar to that used by \cite{grahamdriver2007}, 
but they unexpectedly failed to recover a strong $M_{\rm BH} - n_{\rm sph}$ relation. 
This issue was tackled by \cite{savorgnan2013}, who collected the S\'ersic index measurements published by 
\cite{grahamdriver2007}, \cite{sani2011}, \cite{vika2012} and \cite{beifiori2012} for a sample of 54 galaxies, 
and showed that, by rejecting the most discrepant measurements and averaging the remaning ones, 
a strong $M_{\rm BH} - n_{\rm sph}$ relation was retrieved. 
Remarkably, \cite{savorgnan2013} repeated their analysis upon excluding the S\'ersic index measurements of \cite{grahamdriver2007} 
and still regained a significant $M_{\rm BH} - n_{\rm sph}$ relation. 
This was suggesting that the individual galaxy decompositions of \cite{sani2011}, \cite{vika2012} and \cite{beifiori2012} were not accurate, 
i.e.~each individual study obtained ``noisy'' S\'ersic index measurements 
which prevented the recovery of a strong $M_{\rm BH} - n_{\rm sph}$ relation. \\
Motivated by the need of more accurate galaxy decompositions to refine and re-investigate scaling relations 
between the black hole mass and several galaxy structural parameters, 
we performed state-of-the-art modelling for the largest sample of galaxies to date (Savorgnan \& Graham 2015, hereafter \emph{Paper I}),  
for which a dynamical measurement of the black hole mass was available.
In doing so, we used $3.6\rm~\mu m$ \emph{Spitzer} satellite imagery, 
given its superb capability to trace the stellar mass (\citealt{sheth2010}, and references therein). 
In Savorgnan et al.~(2015, hereafter \emph{Paper II}) we examined the correlations between black hole mass and 
galaxy luminosity, spheroid luminosity and spheroid stellar mass. 
Here we focus on the $M_{\rm BH} - n_{\rm sph}$ relation. \\
This paper (\emph{Paper III}) is structured as follows...


\section{Data}
We populated the $L_{\rm sph} - n_{\rm sph}$ and $M_{\rm BH} - n_{\rm sph}$ diagrams 
with the same galaxy sample (Table \ref{tab:sample}) used in \emph{Paper II}, 
i.e.~66 galaxies for which a dynamical measurement of the black hole mass has been reported in the literature 
(by \citealt{grahamscott2013} or \citealt{rusli2013}) 
and for which we were able to successfully model the light distribution and measure the spheroid structural parameters 
using $3.6\rm~\mu m$ \emph{Spitzer} satellite images. 
Our galaxy decompositions take into account bulge, disks, spiral arms, bars, rings, halo, 
extended or unresolved nuclear source and partially depleted core, 
and -- for the first time -- they were checked to be consistent with the galaxy kinematics 
\citep{atlas3dIII,scott2014,arnold2014}. 
Kinematical information was used to confirm the presence of disk components 
in the majority of early-type (elliptical + lenticular) galaxies and, more importantly,  
to estabilish the radial extent of these disks, 
which in most cases is not obvious from a visual inspection of the galaxy images. 
This enabled us to distinguish between intermediate-scale disks, 
that are fully embedded in the spheroid,  
and large-scale disks, that encase the bulge and dominate the light at large radii.  
Savorgnan \& Graham (2015) explain that when an intermediate-scale disk is misclassified and modeled as a large-scale disk, 
the luminosity of the spheroid is underestimated, 
hence the galaxy incorrectly appears as a positive outlier (an ``over-massive'' black hole) in the $M_{\rm BH} - L_{\rm sph}$ diagram. 
A detailed description of the dataset used here, the data reduction process and the galaxy modelling technique that we developed 
can be found in \emph{Paper I}, 
along with a discussion of how we estimated the uncertainties on the spheroid S\'ersic indices\footnote{The uncertainties associated with 
the spheroid S\'ersic indices have been estimated with a method that takes into account systematic errors. 
This method consists in comparing, for each of our galaxies, the measurements of the spheroid S\'ersic index obtained by different studies 
with that obtained by us. 
Systematic errors are typically not considered by popular 2D fitting codes which report only the statistical errors 
associated with their fitted parameters. 
Refer to \emph{Paper I} for a detailed discussion on this. }.  
The morphological classification (E = elliptical; E/S0 = elliptical/lenticular; S0 = lenticular; 
S0/Sp = lenticular/spiral; Sp = spiral; and ``merger'') follows from the galaxy decompositions illustrated in \emph{Paper I}. 
As in \emph{Paper II}, we will refer to early-type galaxies (E+S0) and late-type galaxies (Sp). 
The early-type bin includes the two galaxies classified as E/S0, 
whereas the two galaxies classified as S0/Sp and the two galaxies classified as mergers are included in neither the early- nor the late-type bin.

\begin{table*}                                        
\small                                                
\begin{center}                                        
\caption{Galaxy sample.} 
\begin{tabular}{llllllrll}                           
\tableline                                                
\multicolumn{1}{l}{{\bf Galaxy}} &                   
\multicolumn{1}{l}{{\bf Type}} &                     
\multicolumn{1}{l}{{\bf Core}} &                     
\multicolumn{1}{l}{{\bf Distance}} &                 
\multicolumn{1}{l}{{\bf $\bm{M_{\rm BH}}$}} &  
\multicolumn{1}{l}{{\bf $\bm{MAG_{\rm sph}}$}} &  
\multicolumn{1}{l}{{\bf $\bm{MAG_{\rm gal}}$}} &  
\multicolumn{1}{l}{{\bf $\bm{[3.6]-[4.5]}$}} &  
\multicolumn{1}{l}{{\bf $\bm{M_{\rm *,sph}}$}} \\  
\multicolumn{1}{l}{} &                                
\multicolumn{1}{l}{} &                                
\multicolumn{1}{l}{} &                                
\multicolumn{1}{l}{[Mpc]} &                           
\multicolumn{1}{l}{$[10^8~\rm M_{\odot}]$} &         
\multicolumn{1}{l}{[mag]} &                                
\multicolumn{1}{l}{[mag]} &                                
\multicolumn{1}{l}{[mag]} &                                
\multicolumn{1}{l}{$[10^{10}~\rm M_{\odot}]$} \\                             
\multicolumn{1}{l}{(1)} &                             
\multicolumn{1}{l}{(2)} &                             
\multicolumn{1}{l}{(3)} &                             
\multicolumn{1}{l}{(4)} &                             
\multicolumn{1}{l}{(5)} &                             
\multicolumn{1}{l}{(6)} &                             
\multicolumn{1}{l}{(7)} &                             
\multicolumn{1}{l}{(8)} &                             
\multicolumn{1}{l}{(9)} \\                         
\tableline                                                
IC 1459  &  E  &  yes   &  $28.4$  &  $24_{-10}^{+10}$   &  $-26.15_{-0.11}^{+0.18}$   &  $-26.15 \pm 0.25$ 
 &  $-0.12$  &  $27_{-23}^{+30}$   \\ 
IC 2560  &  Sp (bar)  &  no?  &  $40.7$  &  $0.044_{-0.022}^{+0.044}$   &  $-22.27_{-0.58}^{+0.66}$   &  $-24.76 \pm 0.25$ 
 &  $-0.08$  &  $1.0_{-0.6}^{+1.8}$   \\ 
IC 4296  &  E  &  yes?  &  $40.7$  &  $11_{-2}^{+2}$   &  $-26.35_{-0.11}^{+0.18}$   &  $-26.35 \pm 0.25$ 
 &  $-0.12$  &  $31_{-26}^{+34}$   \\ 
M104  &  S0/Sp  &  yes   &  $9.5$  &  $6.4_{-0.4}^{+0.4}$   &  $-23.91_{-0.58}^{+0.66}$   &  $-25.21 \pm 0.25$ 
 &  $-0.12$  &  $3.4_{-1.9}^{+5.8}$   \\ 
M105  &  E  &  yes   &  $10.3$  &  $4_{-1}^{+1}$   &  $-24.29_{-0.58}^{+0.66}$   &  $-24.29 \pm 0.25$ 
 &  $-0.10$  &  $5.6_{-3.0}^{+9.5}$   \\ 
M106  &  Sp (bar)  &  no   &  $7.2$  &  $0.39_{-0.01}^{+0.01}$   &  $-21.11_{-0.11}^{+0.18}$   &  $-24.04 \pm 0.25$ 
 &  $-0.08$  &  $0.37_{-0.31}^{+0.41}$   \\ 
M31  &  Sp (bar)  &  no   &  $0.7$  &  $1.4_{-0.3}^{+0.9}$   &  $-22.74_{-0.11}^{+0.18}$   &  $-24.67 \pm 0.25$ 
 &  $-0.09$  &  $1.5_{-1.3}^{+1.6}$   \\ 
M49  &  E  &  yes   &  $17.1$  &  $25_{-1}^{+3}$   &  $-26.54_{-0.11}^{+0.18}$   &  $-26.54 \pm 0.25$ 
 &  $-0.12$  &  $39_{-33}^{+43}$   \\ 
M59  &  E  &  no   &  $17.8$  &  $3.9_{-0.4}^{+0.4}$   &  $-25.18_{-0.11}^{+0.18}$   &  $-25.27 \pm 0.25$ 
 &  $-0.09$  &  $14_{-11}^{+15}$   \\ 
M64  &  Sp  &  no?  &  $7.3$  &  $0.016_{-0.004}^{+0.004}$   &  $-21.54_{-0.11}^{+0.18}$   &  $-24.24 \pm 0.25$ 
 &  $-0.06$  &  $0.64_{-0.55}^{+0.71}$   \\ 
M81  &  Sp (bar)  &  no   &  $3.8$  &  $0.74_{-0.11}^{+0.21}$   &  $-23.01_{-0.66}^{+0.88}$   &  $-24.43 \pm 0.25$ 
 &  $-0.09$  &  $1.9_{-0.9}^{+3.6}$   \\ 
M84  &  E  &  yes   &  $17.9$  &  $9.0_{-0.8}^{+0.9}$   &  $-26.01_{-0.58}^{+0.66}$   &  $-26.01 \pm 0.25$ 
 &  $-0.10$  &  $28_{-15}^{+47}$   \\ 
M87  &  E  &  yes   &  $15.6$  &  $58.0_{-3.5}^{+3.5}$   &  $-26.00_{-0.58}^{+0.66}$   &  $-26.00 \pm 0.25$ 
 &  $-0.11$  &  $26_{-14}^{+44}$   \\ 
M89  &  E  &  yes   &  $14.9$  &  $4.7_{-0.5}^{+0.5}$   &  $-24.48_{-0.58}^{+0.66}$   &  $-24.74 \pm 0.25$ 
 &  $-0.11$  &  $6.3_{-3.4}^{+10.7}$   \\ 
M94  &  Sp (bar)  &  no?  &  $4.4$  &  $0.060_{-0.014}^{+0.014}$   &  $-22.08_{-0.11}^{+0.18}$   &  $\leq-23.36$   &  $-0.07$  &  $1.00_{-0.85}^{+1.11}$   \\ 
M96  &  Sp (bar)  &  no   &  $10.1$  &  $0.073_{-0.015}^{+0.015}$   &  $-22.15_{-0.11}^{+0.18}$   &  $-24.20 \pm 0.25$ 
 &  $-0.08$  &  $0.97_{-0.83}^{+1.08}$   \\ 
NGC 0524  &  S0  &  yes   &  $23.3$  &  $8.3_{-1.3}^{+2.7}$   &  $-23.19_{-0.11}^{+0.18}$   &  $-24.92 \pm 0.25$ 
 &  $-0.09$  &  $2.2_{-1.9}^{+2.5}$   \\ 
NGC 0821  &  E  &  no   &  $23.4$  &  $0.39_{-0.09}^{+0.26}$   &  $-24.00_{-0.66}^{+0.88}$   &  $-24.26 \pm 0.25$ 
 &  $-0.09$  &  $4.7_{-2.1}^{+8.7}$   \\ 
NGC 1023  &  S0 (bar)  &  no   &  $11.1$  &  $0.42_{-0.04}^{+0.04}$   &  $-22.82_{-0.11}^{+0.18}$   &  $-24.20 \pm 0.25$ 
 &  $-0.10$  &  $1.5_{-1.3}^{+1.7}$   \\ 
NGC 1300  &  Sp (bar)  &  no   &  $20.7$  &  $0.73_{-0.35}^{+0.69}$   &  $-22.06_{-0.58}^{+0.66}$   &  $-24.16 \pm 0.25$ 
 &  $-0.10$  &  $0.70_{-0.38}^{+1.19}$   \\ 
NGC 1316  &  merger  &  no   &  $18.6$  &  $1.50_{-0.80}^{+0.75}$   &  $-24.89_{-0.58}^{+0.66}$   &  $-26.48 \pm 0.25$ 
 &  $-0.10$  &  $9.5_{-5.2}^{+16.2}$   \\ 
NGC 1332  &  E/S0  &  no   &  $22.3$  &  $14_{-2}^{+2}$   &  $-24.89_{-0.66}^{+0.88}$   &  $-24.95 \pm 0.25$ 
 &  $-0.12$  &  $8.2_{-3.6}^{+15.0}$   \\ 
NGC 1374  &  E  &  no?  &  $19.2$  &  $5.8_{-0.5}^{+0.5}$   &  $-23.68_{-0.11}^{+0.18}$   &  $-23.70 \pm 0.25$ 
 &  $-0.09$  &  $3.6_{-3.0}^{+4.0}$   \\ 
NGC 1399  &  E  &  yes   &  $19.4$  &  $4.7_{-0.6}^{+0.6}$   &  $-26.43_{-0.11}^{+0.18}$   &  $-26.46 \pm 0.25$ 
 &  $-0.12$  &  $33_{-28}^{+37}$   \\ 
NGC 2273  &  Sp (bar)  &  no   &  $28.5$  &  $0.083_{-0.004}^{+0.004}$   &  $-23.00_{-0.58}^{+0.66}$   &  $-24.21 \pm 0.25$ 
 &  $-0.08$  &  $2.0_{-1.1}^{+3.4}$   \\ 
NGC 2549  &  S0 (bar)  &  no   &  $12.3$  &  $0.14_{-0.13}^{+0.02}$   &  $-21.25_{-0.11}^{+0.18}$   &  $-22.60 \pm 0.25$ 
 &  $-0.10$  &  $0.35_{-0.30}^{+0.39}$   \\ 
NGC 2778  &  S0 (bar)  &  no   &  $22.3$  &  $0.15_{-0.10}^{+0.09}$   &  $-20.80_{-0.58}^{+0.66}$   &  $-22.44 \pm 0.25$ 
 &  $-0.09$  &  $0.25_{-0.14}^{+0.43}$   \\ 
NGC 2787  &  S0 (bar)  &  no   &  $7.3$  &  $0.40_{-0.05}^{+0.04}$   &  $-20.11_{-0.58}^{+0.66}$   &  $-22.28 \pm 0.25$ 
 &  $-0.10$  &  $0.12_{-0.07}^{+0.20}$   \\ 
NGC 2974  &  Sp (bar)  &  no   &  $20.9$  &  $1.7_{-0.2}^{+0.2}$   &  $-22.95_{-0.58}^{+0.66}$   &  $-24.16 \pm 0.25$ 
 &  $-0.09$  &  $1.8_{-1.0}^{+3.1}$   \\ 
NGC 3079  &  Sp (bar)  &  no?  &  $20.7$  &  $0.024_{-0.012}^{+0.024}$   &  $-23.01_{-0.58}^{+0.66}$   &  $\leq-24.45$   &  $-0.07$  &  $2.4_{-1.3}^{+4.0}$   \\ 
NGC 3091  &  E  &  yes   &  $51.2$  &  $36_{-2}^{+1}$   &  $-26.28_{-0.11}^{+0.18}$   &  $-26.28 \pm 0.25$ 
 &  $-0.12$  &  $30_{-26}^{+34}$   \\ 
NGC 3115  &  E/S0  &  no   &  $9.4$  &  $8.8_{-2.7}^{+10.0}$   &  $-24.22_{-0.11}^{+0.18}$   &  $-24.40 \pm 0.25$ 
 &  $-0.11$  &  $4.9_{-4.1}^{+5.4}$   \\ 
NGC 3227  &  Sp (bar)  &  no   &  $20.3$  &  $0.14_{-0.06}^{+0.10}$   &  $-21.76_{-0.58}^{+0.66}$   &  $-24.26 \pm 0.25$ 
 &  $-0.08$  &  $0.67_{-0.37}^{+1.15}$   \\ 
NGC 3245  &  S0 (bar)  &  no   &  $20.3$  &  $2.0_{-0.5}^{+0.5}$   &  $-22.43_{-0.11}^{+0.18}$   &  $-23.88 \pm 0.25$ 
 &  $-0.10$  &  $1.0_{-0.9}^{+1.1}$   \\ 
NGC 3377  &  E  &  no   &  $10.9$  &  $0.77_{-0.06}^{+0.04}$   &  $-23.49_{-0.58}^{+0.66}$   &  $-23.57 \pm 0.25$ 
 &  $-0.06$  &  $4.0_{-2.2}^{+6.8}$   \\ 
NGC 3384  &  S0 (bar)  &  no   &  $11.3$  &  $0.17_{-0.02}^{+0.01}$   &  $-22.43_{-0.11}^{+0.18}$   &  $-23.74 \pm 0.25$ 
 &  $-0.08$  &  $1.2_{-1.0}^{+1.3}$   \\ 
NGC 3393  &  Sp (bar)  &  no   &  $55.2$  &  $0.34_{-0.02}^{+0.02}$   &  $-23.48_{-0.58}^{+0.66}$   &  $-25.29 \pm 0.25$ 
 &  $-0.10$  &  $2.8_{-1.5}^{+4.7}$   \\ 
NGC 3414  &  E  &  no   &  $24.5$  &  $2.4_{-0.3}^{+0.3}$   &  $-24.35_{-0.11}^{+0.18}$   &  $-24.42 \pm 0.25$ 
 &  $-0.09$  &  $6.5_{-5.5}^{+7.2}$   \\ 
NGC 3489  &  S0/Sp (bar)  &  no   &  $11.7$  &  $0.058_{-0.008}^{+0.008}$   &  $-21.13_{-0.58}^{+0.66}$   &  $-23.07 \pm 0.25$ 
 &  $-0.06$  &  $0.42_{-0.23}^{+0.72}$   \\ 
NGC 3585  &  E  &  no   &  $19.5$  &  $3.1_{-0.6}^{+1.4}$   &  $-25.52_{-0.58}^{+0.66}$   &  $-25.55 \pm 0.25$ 
 &  $-0.10$  &  $18_{-10}^{+30}$   \\ 
NGC 3607  &  E  &  no   &  $22.2$  &  $1.3_{-0.5}^{+0.5}$   &  $-25.36_{-0.58}^{+0.66}$   &  $-25.45 \pm 0.25$ 
 &  $-0.10$  &  $15_{-8}^{+25}$   \\ 
NGC 3608  &  E  &  yes   &  $22.3$  &  $2.0_{-0.6}^{+1.1}$   &  $-24.50_{-0.58}^{+0.66}$   &  $-24.50 \pm 0.25$ 
 &  $-0.08$  &  $7.8_{-4.3}^{+13.4}$   \\ 
NGC 3842  &  E  &  yes   &  $98.4$  &  $97_{-26}^{+30}$   &  $-27.00_{-0.11}^{+0.18}$   &  $-27.04 \pm 0.25$ 
 &  $-0.11$  &  $61_{-52}^{+68}$   \\ 
NGC 3998  &  S0 (bar)  &  no   &  $13.7$  &  $8.1_{-1.9}^{+2.0}$   &  $-22.32_{-0.66}^{+0.88}$   &  $-23.53 \pm 0.25$ 
 &  $-0.12$  &  $0.78_{-0.35}^{+1.43}$   \\ 
NGC 4026  &  S0 (bar)  &  no   &  $13.2$  &  $1.8_{-0.3}^{+0.6}$   &  $-21.58_{-0.66}^{+0.88}$   &  $-23.16 \pm 0.25$ 
 &  $-0.09$  &  $0.50_{-0.22}^{+0.92}$   \\ 
NGC 4151  &  Sp (bar)  &  no   &  $20.0$  &  $0.65_{-0.07}^{+0.07}$   &  $-23.40_{-0.58}^{+0.66}$   &  $-24.44 \pm 0.25$ 
 &  $-0.09$  &  $2.8_{-1.5}^{+4.8}$   \\ 
NGC 4261  &  E  &  yes   &  $30.8$  &  $5_{-1}^{+1}$   &  $-25.72_{-0.58}^{+0.66}$   &  $-25.76 \pm 0.25$ 
 &  $-0.12$  &  $18_{-10}^{+30}$   \\ 
NGC 4291  &  E  &  yes   &  $25.5$  &  $3.3_{-2.5}^{+0.9}$   &  $-24.05_{-0.58}^{+0.66}$   &  $-24.05 \pm 0.25$ 
 &  $-0.11$  &  $3.9_{-2.1}^{+6.7}$   \\ 
NGC 4388  &  Sp (bar)  &  no?  &  $17.0$  &  $0.075_{-0.002}^{+0.002}$   &  $-21.26_{-0.66}^{+0.88}$   &  $\leq-23.50$   &  $-0.07$  &  $0.46_{-0.21}^{+0.85}$   \\ 
NGC 4459  &  S0  &  no   &  $15.7$  &  $0.68_{-0.13}^{+0.13}$   &  $-23.48_{-0.58}^{+0.66}$   &  $-24.01 \pm 0.25$ 
 &  $-0.09$  &  $2.9_{-1.6}^{+5.0}$   \\ 
NGC 4473  &  E  &  no   &  $15.3$  &  $1.2_{-0.9}^{+0.4}$   &  $-23.88_{-0.58}^{+0.66}$   &  $-24.11 \pm 0.25$ 
 &  $-0.10$  &  $3.9_{-2.1}^{+6.6}$   \\ 
NGC 4564  &  S0  &  no   &  $14.6$  &  $0.60_{-0.09}^{+0.03}$   &  $-22.30_{-0.11}^{+0.18}$   &  $-22.99 \pm 0.25$ 
 &  $-0.11$  &  $0.82_{-0.70}^{+0.91}$   \\ 
NGC 4596  &  S0 (bar)  &  no   &  $17.0$  &  $0.79_{-0.33}^{+0.38}$   &  $-22.73_{-0.11}^{+0.18}$   &  $-24.18 \pm 0.25$ 
 &  $-0.08$  &  $1.6_{-1.3}^{+1.7}$   \\ 
\tableline         
\end{tabular}   
\label{tab:sample} 
\end{center}    
\end{table*}    

\begin{table*}                                        
\small                                                
\begin{center}                                        
\begin{tabular}{llllllrll}                           
\tableline                                                
\multicolumn{1}{l}{{\bf Galaxy}} &                   
\multicolumn{1}{l}{{\bf Type}} &                     
\multicolumn{1}{l}{{\bf Core}} &                     
\multicolumn{1}{l}{{\bf Distance}} &                 
\multicolumn{1}{l}{{\bf $\bm{M_{\rm BH}}$}} &  
\multicolumn{1}{l}{{\bf $\bm{MAG_{\rm sph}}$}} &  
\multicolumn{1}{l}{{\bf $\bm{MAG_{\rm gal}}$}} &  
\multicolumn{1}{l}{{\bf $\bm{[3.6]-[4.5]}$}} &  
\multicolumn{1}{l}{{\bf $\bm{M_{\rm *,sph}}$}} \\  
\multicolumn{1}{l}{} &                                
\multicolumn{1}{l}{} &                                
\multicolumn{1}{l}{} &                                
\multicolumn{1}{l}{[Mpc]} &                           
\multicolumn{1}{l}{$[10^8~\rm M_{\odot}]$} &         
\multicolumn{1}{l}{[mag]} &                                
\multicolumn{1}{l}{[mag]} &                                
\multicolumn{1}{l}{[mag]} &                                
\multicolumn{1}{l}{$[10^{10}~\rm M_{\odot}]$} \\                             
\multicolumn{1}{l}{(1)} &                             
\multicolumn{1}{l}{(2)} &                             
\multicolumn{1}{l}{(3)} &                             
\multicolumn{1}{l}{(4)} &                             
\multicolumn{1}{l}{(5)} &                             
\multicolumn{1}{l}{(6)} &                             
\multicolumn{1}{l}{(7)} &                             
\multicolumn{1}{l}{(8)} &                             
\multicolumn{1}{l}{(9)} \\                         
\tableline                                                
NGC 4697  &  E  &  no   &  $11.4$  &  $1.8_{-0.1}^{+0.2}$   &  $-24.82_{-0.66}^{+0.88}$   &  $-24.94 \pm 0.25$ 
 &  $-0.09$  &  $10_{-4}^{+18}$   \\ 
NGC 4889  &  E  &  yes   &  $103.2$  &  $210_{-160}^{+160}$   &  $-27.54_{-0.11}^{+0.18}$   &  $-27.54 \pm 0.25$ 
 &  $-0.12$  &  $91_{-77}^{+101}$   \\ 
NGC 4945  &  Sp (bar)  &  no?  &  $3.8$  &  $0.014_{-0.007}^{+0.014}$   &  $-20.96_{-0.58}^{+0.66}$   &  $\leq-23.79$   &  $-0.06$  &  $0.36_{-0.20}^{+0.62}$   \\ 
NGC 5077  &  E  &  yes   &  $41.2$  &  $7.4_{-3.0}^{+4.7}$   &  $-25.45_{-0.11}^{+0.18}$   &  $-25.45 \pm 0.25$ 
 &  $-0.11$  &  $15_{-13}^{+17}$   \\ 
NGC 5128  &  merger  &  no?  &  $3.8$  &  $0.45_{-0.10}^{+0.17}$   &  $-23.89_{-0.66}^{+0.88}$   &  $-24.97 \pm 0.25$ 
 &  $-0.07$  &  $5.0_{-2.2}^{+9.1}$   \\ 
NGC 5576  &  E  &  no   &  $24.8$  &  $1.6_{-0.4}^{+0.3}$   &  $-24.44_{-0.11}^{+0.18}$   &  $-24.44 \pm 0.25$ 
 &  $-0.09$  &  $7.1_{-6.0}^{+7.9}$   \\ 
NGC 5845  &  S0  &  no   &  $25.2$  &  $2.6_{-1.5}^{+0.4}$   &  $-22.96_{-0.66}^{+0.88}$   &  $-23.10 \pm 0.25$ 
 &  $-0.12$  &  $1.4_{-0.6}^{+2.6}$   \\ 
NGC 5846  &  E  &  yes   &  $24.2$  &  $11_{-1}^{+1}$   &  $-25.81_{-0.58}^{+0.66}$   &  $-25.81 \pm 0.25$ 
 &  $-0.10$  &  $22_{-12}^{+38}$   \\ 
NGC 6251  &  E  &  yes?  &  $104.6$  &  $5_{-2}^{+2}$   &  $-26.75_{-0.11}^{+0.18}$   &  $-26.75 \pm 0.25$ 
 &  $-0.12$  &  $46_{-39}^{+51}$   \\ 
NGC 7052  &  E  &  yes   &  $66.4$  &  $3.7_{-1.5}^{+2.6}$   &  $-26.32_{-0.11}^{+0.18}$   &  $-26.32 \pm 0.25$ 
 &  $-0.11$  &  $33_{-28}^{+36}$   \\ 
NGC 7619  &  E  &  yes   &  $51.5$  &  $25_{-3}^{+8}$   &  $-26.35_{-0.58}^{+0.66}$   &  $-26.41 \pm 0.25$ 
 &  $-0.11$  &  $33_{-18}^{+56}$   \\ 
NGC 7768  &  E  &  yes   &  $112.8$  &  $13_{-4}^{+5}$   &  $-26.90_{-0.58}^{+0.66}$   &  $-26.90 \pm 0.25$ 
 &  $-0.11$  &  $57_{-31}^{+98}$   \\ 
UGC 03789  &  Sp (bar)  &  no?  &  $48.4$  &  $0.108_{-0.005}^{+0.005}$   &  $-22.77_{-0.66}^{+0.88}$   &  $-24.20 \pm 0.25$ 
 &  $-0.07$  &  $1.9_{-0.8}^{+3.4}$   \\ 
\tableline         
\end{tabular}   
\tablecomments{\emph{Column (1)}: Galaxy name. 
\emph{Column (2)}: Morphological type (E=elliptical, S0=lenticular, Sp=spiral, merger). 		       The morphological classification of four galaxies is uncertain (E/S0 or S0/Sp). 		       The presence of a bar is indicated. 
\emph{Column (3)}: Presence of a partially depleted core. 			The question mark is used when the classification has come from the velocity dispersion criteria mentioned in Section \ref{sec:data}. 
\emph{Column (4)}: Distance. 
\emph{Column (5)}: Black hole mass. 
\emph{Column (6)}: Absolute $3.6\rm~\mu m$ bulge magnitude. 		       Bulge magnitudes come from our state-of-the-art multicomponent galaxy decompositions (\emph{Paper I}), 		       which include bulges, disks, bars, spiral arms, rings, haloes, extended or unresolved nuclear sources and partially depleted cores,                        and that -- for the first time -- were checked to be consistent with the galaxy kinematics. 		       The uncertainties were estimated with a method that takes into account systematic errors, which are typically not considered by popular 2D fitting codes. 
\emph{Column (7)}: Absolute $3.6\rm~\mu m$ galaxy magnitude. 			Four galaxies had their magnitudes overestimated, which are give here as upper limits. 
\emph{Column (8)}: $[3.6]-[4.5]$ colour. 
\emph{Column (9)}: Bulge stellar mass. } 
\end{center}    
\end{table*}    


\section{Analysis and results}
As in \emph{Paper II}, a linear regression analysis of the $L_{\rm sph} - n_{\rm sph}$ 
(Table \ref{tab:lregLn} and Figure \ref{fig:magn})
and $M_{\rm BH} - n_{\rm sph}$ (Table \ref{tab:lregMn} and Figure \ref{fig:mbhn}) diagrams 
was performed using three different routines: 
the BCES code from \cite{akritasbershady1996}, 
the FITEXY routine \citep{press1992}, as modified by \cite{tremaine2002}, 
and the Bayesian estimator {\tt linmix\_err} \citep{linmixerr}.
All of these three routines take into account the intrinsic scatter, 
but only the FITEXY and the {\tt linmix\_err} codes allow one to quantify it.
We report both symmetrical and nonsymmetrical linear regressions.  
Symmetrical regressions are meant to be compared with theoretical expectations, 
whereas nonsymmetrical forward ($Y|X$) regressions -- which minimize the scatter in the $Y$ direction -- 
allow one to predict the value of the observable $Y$ with the best possible precision. \\
We searched for extreme outliers in both the $L_{\rm sph} - n_{\rm sph}$ and $M_{\rm BH} - n_{\rm sph}$ diagrams, 
and found that in our $L_{\rm sph} - n_{\rm sph}$ plot there are no $3\sigma$ outliers, 
whereas in our $M_{\rm BH} - n_{\rm sph}$ plot 
the S0 galaxies NGC 0524 and NGC 3998 reside more than $3\sigma$ from the bisector linear regression for all galaxies. 
These two galaxies have therefore been excluded from the rest of the analysis. \\
{\bf fitexy does better job, we report that} \\


\subsection{$L_{\rm sph} - n_{\rm sph}$}
Following \cite{graham2001}, 
who showed that the $L_{\rm sph} - n_{\rm sph}$ relation is different for elliptical galaxies and the bulges of disk galaxies (S0+Sp), 
\cite{savorgnan2013} re-analyzed the data from \cite{grahamguzman2003} and \cite{graham2013} 
and obtained two separate $L_{\rm sph} - n_{\rm sph}$ symmetrical linear regressions for elliptical galaxies and the bulges of disk galaxies 
(in the B- and K-band, respectively). 
At the time, the $L_{\rm sph} - n_{\rm sph}$ datasets from \cite{grahamguzman2003} and \cite{graham2013} were of the best quality available 
to investigate the $L_{\rm sph} - n_{\rm sph}$ relation for different galaxy morphological types. 
However, these datasets were not the output of a homogeneous analysis, 
but the collection of the results from various past bulge/disk decomposition studies. 
It is also worth mentioning that, 
although the K-band luminosities of the bulges of disk galaxies had been corrected for internal dust absorption, 
{\bf the dust corrections are not precise today refs??, imagine at the time.....}  
Here we re-investigate the $L_{\rm sph} - n_{\rm sph}$ diagram with our top-quality dataset  
derived from our state-of-the-art multicomponent galaxy decompositions, 
that were performed in a consistent manner 
and using the $3.6\rm~\mu m$ band, which is less affected by dust extinction than the K-band. \\
The values of the slope and intercept of the bisector linear regression for the lenticular galaxies 
are not consistent within the errors with those for the spiral galaxies, 
but are consistent within the errors with those for the elliptical galaxies.  
Given this, we conclude that in the $L_{\rm sph} - n_{\rm sph}$ diagram 
elliptical and lenticular galaxies form together a single sequence, 
whereas the combination of lenticular and spiral galaxies do not. 
{\bf why?? dust? multicomponent? classification?}
 

ell and s0 have same slope -> early seq \\
different from late seq \\


\begin{figure}[h]
\begin{center}
\includegraphics[width=\columnwidth]{images/mag_vs_n_maj.pdf}
\caption{Spheroid absolute magnitude (at $3.6\rm~\mu m$) plotted against spheroid S\'ersic index 
measured along the galaxy major-axis. 
Symbols are coded according to the galaxy morphological type (see legend). 
The red dashed line indicates the FITEXY bisector linear regression for the $45-2=43$  early-type galaxies (E+S0), 
with the red shaded area denoting its $1\sigma$ uncertainty. 
The shallower blue solid line shows the FITEXY bisector linear regression for the bulges of the 17 late-type (Sp) galaxies, 
with the blue shaded area denoting its $1\sigma$ uncertainty. 
The inset panel shows the error bars associated to each data point,  
where the error bars have the same color coding as the symbols in the main figure. 
}
\label{fig:magn}
\end{center}
\end{figure}

\subsection{$M_{\rm BH} - n_{\rm sph}$}
in sav13 we had m-l from gs13 for core sersic and sersic \\
however in paper II we say that this is not the case, there are red and blue seq \\
therefore we look for red and blue seq here \\
consistent with each other \\



\begin{figure}[h]
\begin{center}
\includegraphics[width=\columnwidth]{images/mbh_vs_n_maj.pdf}
\caption{Black hole mass plotted against spheroid S\'ersic index measured along the galaxy major-axis. 
Symbols are coded according to the galaxy morphological type (see legend). 
The red dashed line indicates the FITEXY bisector linear regression for the $45-2=43$  early-type galaxies (E+S0), 
with the red shaded area denoting its $1\sigma$ uncertainty. 
The shallower blue solid line shows the FITEXY bisector linear regression for the bulges of the 17 late-type (Sp) galaxies, 
with the blue shaded area denoting its $1\sigma$ uncertainty. 
The inset panel shows the error bars associated to each data point,  
where the error bars have the same color coding as the symbols in the main figure. 
}
\label{fig:mbhn}
\end{center}
\end{figure}

\section{Discussion}
intr scatter of correlation for all?\\
data suggest that m-n more fundamental than l-n ? \\



\begin{table*}
\centering
\caption{Linear regression analysis of the $L_{\rm sph} - n_{\rm sph}$ diagram.}
\begin{tabular}{llccccc}
\tableline
\tableline
{\bf Subsample (size)} & {\bf Regression} & $\boldsymbol \alpha$ & $\boldsymbol \beta$ & $\boldsymbol \langle \log n_{\rm sph} \rangle$ & $\boldsymbol \epsilon$ & $\boldsymbol \Delta$ \\ 
\tableline 
\\
 & \multicolumn{6}{l}{$MAG_{\rm sph}/{\rm [mag]} = \alpha + \beta \bigl(\log n_{\rm sph} - \langle \log n_{\rm sph,maj} \rangle \bigr)$} \\ [0.5em]
All (62)               & BCES $(Y|X)$               & $-23.88 \pm 0.15$ & $-7.17 \pm 0.80$ & $0.51$ & $-$ & $1.18$ \\
                       & mFITEXY $(Y|X)$            & $-23.95 \pm 0.13$ & $-6.70 \pm 0.45$ & $0.51$ & $0.56^{+0.15}_{-0.10}$ & $0.98$ \\
                       & {\tt linmix\_err} $(Y|X)$  & $-23.92 \pm 0.15$ & $-6.40 \pm 0.57$ & $0.51$ & $0.74 \pm 0.13$ & $1.07$ \\ [0.5em]
                       & BCES $(X|Y)$               & $-23.88 \pm 0.14$ & $-6.70 \pm 0.51$ & $0.51$ & $-$ & $1.11$ \\
                       & mFITEXY $(X|Y)$            & $-23.94 \pm 0.14$ & $-7.50 \pm 0.52$ & $0.51$ & $0.59^{+0.17}_{-0.11}$ & $1.23$ \\
                       & {\tt linmix\_err} $(X|Y)$  & $-23.94 \pm 0.16$ & $-7.51 \pm 0.62$ & $0.51$ & $0.81 \pm 0.16$ & $1.23$ \\ [0.5em]
                       & BCES Bisector              & $-23.88 \pm 0.14$ & $-6.93 \pm 0.60$ & $0.51$ & $-$ & $1.14$ \\
                       & mFITEXY Bisector           & $-23.94 \pm 0.13$ & $-7.08 \pm 0.34$ & $0.51$ & $-$ & $1.16$ \\
                       & {\tt linmix\_err} Bisector & $-23.93 \pm 0.16$ & $-6.91 \pm 0.42$ & $0.51$ & $-$ & $1.14$ \\ [0.5em]

Elliptical (E) (30)    & BCES $(Y|X)$		    & $-25.46 \pm 1.12$ & $38.47 \pm 114.45$ & $0.76$ & $-$ & $6.37$ \\
		       & mFITEXY $(Y|X)$	    & $-25.74 \pm 0.18$ & $-9.74 \pm 1.59$ & $0.76$ & $0.24^{+0.32}_{-0.24}$ & $0.94$ \\
		       & {\tt linmix\_err} $(Y|X)$  & $-25.65 \pm 0.21$ & $-7.87 \pm 2.15$ & $0.76$ & $0.61 \pm 0.22$ & $1.06$ \\ [0.5em]
		       & BCES $(X|Y)$		    & $-25.46 \pm 0.23$ & $-10.73 \pm 3.21$ & $0.76$ & $-$ & $1.29$ \\
		       & mFITEXY $(X|Y)$	    & $-25.74 \pm 0.20$ & $-10.42 \pm 1.79$ & $0.76$ & $0.22^{+0.38}_{-0.22}$ & $1.29$ \\
		       & {\tt linmix\_err} $(X|Y)$  & $-25.72 \pm 0.28$ & $-10.92 \pm 2.70$ & $0.76$ & $0.73 \pm 0.34$ & $1.33$ \\ [0.5em]
		       & BCES Bisector  	    & $-25.46 \pm 0.20$ & $0.03 \pm 0.05$ & $0.76$ & $-$ & $1.14$ \\
		       & mFITEXY Bisector	    & $-25.74 \pm 0.19$ & $-10.07 \pm 1.19$ & $0.76$ & $-$ & $1.26$ \\
		       & {\tt linmix\_err} Bisector & $-25.68 \pm 0.25$ & $-9.15 \pm 1.74$ & $0.76$ & $-$ & $1.16$ \\ [0.5em]

Lenticular (S0) (11)   & BCES $(Y|X)$		    & $-22.08 \pm 1.66$ & $33.52 \pm 98.87$ & $0.33$ & $-$ & $6.09$ \\
		       & mFITEXY $(Y|X)$	    & $-22.11 \pm 0.24$ & $-6.31 \pm 2.45$ & $0.33$ & $0.42^{+0.28}_{-0.17}$ & $0.71$ \\
		       & {\tt linmix\_err} $(Y|X)$  & $$ & $$ & $0.33$ & $$ & $$ \\ [0.5em]
		       & BCES $(X|Y)$		    & $-22.08 \pm 0.19$ & $-6.83 \pm 1.16$ & $0.33$ & $-$ & $0.71$ \\
		       & mFITEXY $(X|Y)$	    & $-21.94 \pm 0.44$ & $-13.16 \pm 7.91$ & $0.33$ & $0.61^{+0.60}_{-0.56}$ & $1.39$ \\
		       & {\tt linmix\_err} $(X|Y)$  & $$ & $$ & $0.33$ & $$ & $$ \\ [0.5em]
		       & BCES Bisector  	    & $-22.08 \pm 0.30$ & $0.06 \pm 0.05$ & $0.33$ & $-$ & $1.09$ \\
		       & mFITEXY Bisector	    & $-22.05 \pm 0.35$ & $-8.55 \pm 2.79$ & $0.33$ & $-$ & $0.84$ \\
		       & {\tt linmix\_err} Bisector & $$ & $$ & $0.33$ & $-$ & $$ \\ [0.5em]

Spiral (Sp) (17)       & BCES $(Y|X)$		    & $-22.33 \pm 0.26$ & $-5.31 \pm 5.83$ & $0.18$ & $-$ & $1.15$ \\
		       & mFITEXY $(Y|X)$	    & $-22.22 \pm 0.19$ & $-2.17 \pm 0.98$ & $0.18$ & $0.53^{+0.24}_{-0.13}$ & $0.72$ \\
		       & {\tt linmix\_err} $(Y|X)$  & $-22.26 \pm 0.24$ & $-1.53 \pm 1.88$ & $0.18$ & $0.71 \pm      0.22$ & $0.78$ \\ [0.5em]
		       & BCES $(X|Y)$		    & $-22.33 \pm 0.26$ & $-5.19 \pm 3.77$ & $0.18$ & $-$ & $1.13$ \\
		       & mFITEXY $(X|Y)$	    & $-22.28 \pm 0.44$ & $-9.08 \pm 5.31$ & $0.51$ & $1.12^{+0.54}_{-0.31}$ & $1.83$ \\
		       & {\tt linmix\_err} $(X|Y)$  & $-22.24 \pm 0.71$ & $-11.12 \pm 13.59$ & $0.18$ & $1.95 \pm 2.47$ & $2.24$ \\ [0.5em]
		       & BCES Bisector  	    & $-22.33 \pm 0.26$ & $-5.25 \pm 3.38$ & $0.18$ & $-$ & $1.14$ \\
		       & mFITEXY Bisector	    & $-22.23 \pm 0.33$ & $-3.60 \pm 1.29$ & $0.18$ & $-$ & $0.92$ \\
		       & {\tt linmix\_err} Bisector & $-22.25 \pm 0.53$ & $-2.88 \pm 2.66$ & $0.18$ & $-$ & $0.84$ \\ [0.5em]

\tableline 
\tableline
\end{tabular}
\end{table*}

\begin{table*}
\centering
\caption{Linear regression analysis of the $L_{\rm sph} - n_{\rm sph}$ diagram.}
\begin{tabular}{llccccc}
\tableline
\tableline
{\bf Subsample (size)} & {\bf Regression} & $\boldsymbol \alpha$ & $\boldsymbol \beta$ & $\boldsymbol \langle \log n_{\rm sph} \rangle$ & $\boldsymbol \epsilon$ & $\boldsymbol \Delta$ \\ 
\tableline 
\\
Early-type (E+S0) (43) & BCES $(Y|X)$		    & $-24.55 \pm 0.22$ & $-11.84 \pm 2.29$ & $0.64$ & $-$ & $1.50$ \\
		       & mFITEXY $(Y|X)$	    & $-24.74 \pm 0.14$ & $-8.86 \pm 0.66$ & $0.51$ & $0.27^{+0.20}_{-0.27}$ & $0.87$ \\
		       & {\tt linmix\_err} $(Y|X)$  & $-24.70 \pm 0.17$ & $-8.28 \pm 0.87$ & $0.64$ & $0.58 \pm 0.17$ & $0.98$ \\ [0.5em]
		       & BCES $(X|Y)$		    & $-24.55 \pm 0.14$ & $-8.25 \pm 0.63$ & $0.64$ & $-$ & $0.96$ \\
		       & mFITEXY $(X|Y)$	    & $-24.74 \pm 0.14$ & $-9.13 \pm 0.68$ & $0.64$ & $0.23^{+0.25}_{-0.23}$ & $1.08$ \\
		       & {\tt linmix\_err} $(X|Y)$  & $-24.73 \pm 0.18$ & $-9.08 \pm 0.87$ & $0.64$ & $0.60 \pm 0.21$ & $1.07$ \\ [0.5em]
		       & BCES Bisector  	    & $-24.55 \pm 0.17$ & $-9.73 \pm 1.05$ & $0.64$ & $-$ & $1.14$ \\
		       & mFITEXY Bisector	    & $-24.74 \pm 0.14$ & $-8.99 \pm 0.48$ & $0.64$ & $-$ & $1.06$ \\
		       & {\tt linmix\_err} Bisector & $-24.72 \pm 0.17$ & $-8.66 \pm 0.63$ & $0.64$ & $-$ & $1.02$ \\ [0.5em]

Bulge (S0+Sp) (30)     & BCES $(Y|X)$		    & $-22.25 \pm 0.20$ & $-5.88 \pm 3.06$ & $0.26$ & $-$ & $1.16$ \\
		       & mFITEXY $(Y|X)$	    & $-22.19 \pm 0.14$ & $-2.99 \pm 0.73$ & $0.26$ & $0.52^{+0.18}_{-0.10}$ & $0.75$ \\
		       & {\tt linmix\_err} $(Y|X)$  & $-22.20 \pm 0.17$ & $-2.48 \pm 1.21$ & $0.26$ & $0.67 \pm 0.15$ & $0.83$ \\ [0.5em]
		       & BCES $(X|Y)$		    & $-22.25 \pm 0.20$ & $-5.85 \pm 1.83$ & $0.26$ & $-$ & $1.15$ \\
		       & mFITEXY $(X|Y)$	    & $-22.17 \pm 0.25$ & $-7.65 \pm 2.43$ & $0.26$ & $0.87^{+0.30}_{-0.18}$ & $1.46$ \\
		       & {\tt linmix\_err} $(X|Y)$  & $-22.16 \pm 0.31$ & $-7.80 \pm 3.89$ & $0.26$ & $1.18 \pm 0.65$ & $1.48$ \\ [0.5em]
		       & BCES Bisector  	    & $-22.25 \pm 0.20$ & $-5.87 \pm 2.06$ & $0.26$ & $-$ & $1.16$ \\
		       & mFITEXY Bisector	    & $-22.18 \pm 0.20$ & $-4.34 \pm 0.84$ & $0.26$ & $-$ & $0.96$ \\
		       & {\tt linmix\_err} Bisector & $-22.19 \pm 0.25$ & $-3.83 \pm 1.39$ & $0.26$ & $-$ & $0.91$ \\ [1.0em]

% & \multicolumn{6}{l}{$MAG_{\rm sph}/{\rm [mag]} = \alpha + \beta \bigl(\log n_{\rm sph,eq} - \langle \log n_{\rm sph,eq} \rangle \bigr)$} \\ [0.5em]
%
% All (62)		 & BCES $(Y|X)$ 	      & $-23.88 \pm 0.15$ & $-7.04 \pm 0.89$ & $0.50$ & $-$ & $1.19$ \\
%			 & mFITEXY $(Y|X)$	      & $-23.95 \pm 0.12$ & $-6.43 \pm 0.41$ & $0.50$ & $0.55^{+0.16}_{-0.10}$ & $0.96$ \\
%			 & {\tt linmix\_err} $(Y|X)$  & $-23.91 \pm 0.14$ & $-6.04 \pm 0.54$ & $0.50$ & $0.75 \pm 0.13$ & $1.07$ \\ [0.5em]
%			 & BCES $(X|Y)$ 	      & $-23.88 \pm 0.14$ & $-6.73 \pm 0.48$ & $0.50$ & $-$ & $1.14$ \\
%			 & mFITEXY $(X|Y)$	      & $-23.93 \pm 0.13$ & $-7.13 \pm 0.48$ & $0.50$ & $0.58^{+0.17}_{-0.12}$ & $1.21$ \\
%			 & {\tt linmix\_err} $(X|Y)$  & $-23.93 \pm 0.15$ & $-7.14 \pm 0.57$ & $0.50$ & $0.76 \pm 0.16$ & $1.21$ \\ [0.5em]
%			 & BCES Bisector	      & $-23.88 \pm 0.15$ & $-6.88 \pm 0.62$ & $0.50$ & $-$ & $1.17$ \\
%			 & mFITEXY Bisector	      & $-23.94 \pm 0.13$ & $-6.76 \pm 0.31$ & $0.50$ & $-$ & $1.15$ \\
%			 & {\tt linmix\_err} Bisector & $-23.92 \pm 0.15$ & $-6.55 \pm 0.40$ & $0.50$ & $-$ & $1.12$ \\ [0.5em]
%
%
% Elliptical (30)	 & BCES $(Y|X)$ 	      & $-25.46 \pm 0.26$ & $4.92 \pm 4.72$ & $0.77$ & $-$ & $1.46$ \\
%			 & mFITEXY $(Y|X)$	      & $-25.66 \pm 0.24$ & $-12.65 \pm 2.78$ & $0.77$ & $0.42^{+0.38}_{-0.42}$ & $1.27$ \\
%			 & {\tt linmix\_err} $(Y|X)$  & $-25.55 \pm 0.26$ & $-8.54 \pm 4.68$ & $0.77$ & $0.80 \pm 0.24$ & $1.20$ \\ [0.5em]
%			 & BCES $(X|Y)$ 	      & $-25.46 \pm 0.40$ & $-20.83 \pm 11.31$ & $0.77$ & $-$ & $2.26$ \\
%			 & mFITEXY $(X|Y)$	      & $-25.63 \pm 0.31$ & $-16.13 \pm 4.37$ & $0.77$ & $0.48^{+0.48}_{-0.48}$ & $1.80$ \\
%			 & {\tt linmix\_err} $(X|Y)$  & $-25.63 \pm 0.43$ & $-16.66 \pm 6.50$ & $0.77$ & $1.22 \pm 0.65$ & $1.85$ \\ [0.5em]
%			 & BCES Bisector	      & $-25.46 \pm 0.20$ & $-0.08 \pm 0.10$ & $0.77$ & $-$ & $1.14$ \\
%			 & mFITEXY Bisector	      & $-25.65 \pm 0.28$ & $-14.18 \pm 2.43$ & $0.77$ & $-$ & $1.63$ \\
%			 & {\tt linmix\_err} Bisector & $-25.58 \pm 0.35$ & $-11.30 \pm 4.34$ & $0.77$ & $-$ & $1.38$ \\ [0.5em]
%
%\tableline 
%\tableline
%\end{tabular}
%\end{table*}
%
%\begin{table*}
%\centering
%\caption{Linear regression analysis of the $L_{\rm sph} - n_{\rm sph}$ diagram.}
%\begin{tabular}{llccccc}
%\tableline
%\tableline
%{\bf Subsample (size)} & {\bf Regression} & $\boldsymbol \alpha$ & $\boldsymbol \beta$ & $\boldsymbol \langle \log n_{\rm sph} \rangle$ & $\boldsymbol \epsilon$ & $\boldsymbol \Delta$ \\ 
%\tableline 
%\\
%
% Lenticular (11)	 & BCES $(Y|X)$ 	      & $-22.08 \pm 0.89$ & $16.29 \pm 24.44$ & $0.29$ & $-$ & $3.25$ \\
%			 & mFITEXY $(Y|X)$	      & $-22.16 \pm 0.20$ & $-5.84 \pm 1.83$ & $0.29$ & $0.31^{+0.28}_{-0.20}$ & $0.61$ \\
%			 & {\tt linmix\_err} $(Y|X)$  & $$ & $$ & $0.29$ & $$ & $$ \\ [0.5em]
%			 & BCES $(X|Y)$ 	      & $-22.08 \pm 0.20$ & $-7.40 \pm 1.27$ & $0.29$ & $-$ & $0.72$ \\
%			 & mFITEXY $(X|Y)$	      & $-22.11 \pm 0.28$ & $-8.30 \pm 3.33$ & $0.29$ & $0.40^{+0.33}_{-0.35}$ & $0.77$ \\
%			 & {\tt linmix\_err} $(X|Y)$  & $$ & $$ & $0.29$ & $$ & $$ \\ [0.5em]
%			 & BCES Bisector	      & $-22.08 \pm 0.30$ & $0.04 \pm 0.05$ & $0.29$ & $-$ & $1.08$ \\
%			 & mFITEXY Bisector	      & $-22.14 \pm 0.25$ & $-6.86 \pm 1.70$ & $0.29$ & $-$ & $0.70$ \\
%			 & {\tt linmix\_err} Bisector & $$ & $$ & $0.29$ & $$ & $$ \\ [0.5em]
%
%
% Spiral (17)		 & BCES $(Y|X)$ 	      & $-22.33 \pm 0.62$ & $-14.53 \pm 71.43$ & $0.17$ & $-$ & $2.71$ \\
%			 & mFITEXY $(Y|X)$	      & $-22.37 \pm 0.27$ & $-6.53 \pm 2.84$ & $0.17$ & $0.47^{+0.52}_{-0.39}$ & $0.95$ \\
%			 & {\tt linmix\_err} $(Y|X)$  & $-22.35 \pm 1.23$ & $-3.62 \pm 21.94$ & $0.17$ & $0.63 \pm 0.25$ & $0.93$ \\ [0.5em]
%			 & BCES $(X|Y)$ 	      & $-22.33 \pm 0.35$ & $-7.91 \pm 5.53$ & $0.17$ & $-$ & $1.54$ \\
%			 & mFITEXY $(X|Y)$	      & $-22.74 \pm 0.56$ & $-14.91 \pm 10.03$ & $0.17$ & $0.65^{+1.41}_{-0.65}$ & $2.82$ \\
%			 & {\tt linmix\_err} $(X|Y)$  & $-22.66 \pm 0.88$ & $-16.17 \pm 21.34$ & $0.17$ & $1.47 \pm 2.15$ & $3.04$ \\ [0.5em]
%			 & BCES Bisector	      & $-22.33 \pm 0.44$ & $-10.25 \pm 18.84$ & $0.17$ & $-$ & $1.94$ \\
%			 & mFITEXY Bisector	      & $-22.48 \pm 0.44$ & $-9.10 \pm 3.31$ & $0.17$ & $-$ & $1.75$ \\
%			 & {\tt linmix\_err} Bisector & $-22.41 \pm 1.07$ & $-5.98 \pm 28.65$ & $0.17$ & $-$ & $1.24$ \\ [0.5em]
%
%
% Early-type (43)	 & BCES $(Y|X)$ 	      & $-24.55 \pm 0.22$ & $-10.81 \pm 1.96$ & $0.64$ & $-$ & $1.46$ \\
%			 & mFITEXY $(Y|X)$	      & $-24.70 \pm 0.14$ & $-7.89 \pm 0.63$ & $0.64$ & $0.46^{+0.18}_{-0.14}$ & $0.93$ \\
%			 & {\tt linmix\_err} $(Y|X)$  & $-24.63 \pm 0.18$ & $-7.34 \pm 0.83$ & $0.64$ & $0.75 \pm 0.16$ & $1.02$ \\ [0.5em]
%			 & BCES $(X|Y)$ 	      & $-24.55 \pm 0.16$ & $-8.35 \pm 0.68$ & $0.64$ & $-$ & $1.10$ \\
%			 & mFITEXY $(X|Y)$	      & $-24.69 \pm 0.16$ & $-8.65 \pm 0.73$ & $0.64$ & $0.48^{+0.20}_{-0.15}$ & $1.14$ \\
%			 & {\tt linmix\_err} $(X|Y)$  & $-24.68 \pm 0.20$ & $-8.68 \pm 0.91$ & $0.64$ & $0.77 \pm 0.20$ & $1.14$ \\ [0.5em]
%			 & BCES Bisector	      & $-24.55 \pm 0.18$ & $-9.42 \pm 1.00$ & $0.64$ & $-$ & $1.23$ \\
%			 & mFITEXY Bisector	      & $-24.69 \pm 0.15$ & $-8.25 \pm 0.48$ & $0.64$ & $-$ & $1.10$ \\
%			 & {\tt linmix\_err} Bisector & $-24.65 \pm 0.19$ & $-7.96 \pm 0.62$ & $0.64$ & $-$ & $1.07$ \\ [0.5em]
%
%
% Bulge (30)		 & BCES $(Y|X)$ 	      & $-22.25 \pm 0.50$ & $-16.91 \pm 33.45$ & $0.23$ & $-$ & $2.85$ \\
%			 & mFITEXY $(Y|X)$	      & $-22.21 \pm 0.14$ & $-4.25 \pm 0.94$ & $0.23$ & $0.45^{+0.21}_{-0.13}$ & $0.75$ \\
%			 & {\tt linmix\_err} $(Y|X)$  & $-22.23 \pm 0.17$ & $-3.36 \pm 1.62$ & $0.23$ & $0.61 \pm 0.17$ & $0.87$ \\ [0.5em]
%			 & BCES $(X|Y)$ 	      & $-22.25 \pm 0.23$ & $-7.56 \pm 1.94$ & $0.23$ & $-$ & $1.30$ \\
%			 & mFITEXY $(X|Y)$	      & $-22.23 \pm 0.22$ & $-8.20 \pm 2.44$ & $0.23$ & $0.63^{+0.31}_{-0.19}$ & $1.39$ \\
%			 & {\tt linmix\_err} $(X|Y)$  & $-22.22 \pm 0.28$ & $-8.42 \pm 4.00$ & $0.23$ & $0.96 \pm 0.54$ & $1.43$ \\ [0.5em]
%			 & BCES Bisector	      & $-22.25 \pm 0.31$ & $-10.46 \pm 7.05$ & $0.23$ & $-$ & $1.75$ \\
%			 & mFITEXY Bisector	      & $-22.22 \pm 0.19$ & $-5.62 \pm 0.99$ & $0.23$ & $-$ & $1.05$ \\
%			 & {\tt linmix\_err} Bisector & $-22.23 \pm 0.23$ & $-4.84 \pm 1.75$ & $0.23$ & $-$ & $0.98$ \\ [0.5em]


%All (xx)		& BCES $(Y|X)$  	     & $$ & $$ & $$ & $$ & $$ \\
%			& mFITEXY $(Y|X)$	     & $$ & $$ & $$ & $$ & $$ \\
%			& {\tt linmix\_err} $(Y|X)$  & $$ & $$ & $$ & $$ & $$ \\ [0.5em]
%			& BCES $(X|Y)$  	     & $$ & $$ & $$ & $$ & $$ \\
%			& mFITEXY $(X|Y)$	     & $$ & $$ & $$ & $$ & $$ \\
%			& {\tt linmix\_err} $(X|Y)$  & $$ & $$ & $$ & $$ & $$ \\ [0.5em]
%			& BCES Bisector 	     & $$ & $$ & $$ & $$ & $$ \\
%			& mFITEXY Bisector	     & $$ & $$ & $$ & $$ & $$ \\
%			& {\tt linmix\_err} Bisector & $$ & $$ & $$ & $$ & $$ \\ [0.5em]
%

\tableline 
\tableline
\end{tabular}
\label{tab:lregLn} 
\tablecomments{For each subsample, we indicate $\langle \log n_{\rm sph} \rangle$, its average value of spheroid S\'ersic index. 
In the last two columns, we report $\epsilon$, the intrinsic scatter, and $\Delta$, the total rms scatter in the $L_{\rm sph}$ direction. 
all - mergers - outliers
Both the early- and late-type subsamples do not contain the two galaxies classified as S0/Sp and the two galaxies classified as mergers (45+17=66-2-2). }
\end{table*}


\begin{table*}
\centering
\caption{Linear regression analysis of the $M_{\rm BH} - n_{\rm sph}$ diagram.}
\begin{tabular}{llccccc}
\tableline
\tableline
{\bf Subsample (size)} & {\bf Regression} & $\boldsymbol \alpha$ & $\boldsymbol \beta$ & $\boldsymbol \langle \log n_{\rm sph} \rangle$ & $\boldsymbol \epsilon$ & $\boldsymbol \Delta$ \\ 
\tableline 
\\
 & \multicolumn{6}{l}{$\log \bigl( M_{\rm BH}/{\rm [M_\odot]} \bigr) = \alpha + \beta \bigl(\log n_{\rm sph} - \langle \log n_{\rm sph} \rangle \bigr)$} \\ [0.5em]
 All (62)		& BCES $(Y|X)$  	     & $8.14 \pm 0.08$ & $3.56 \pm 0.38$ & $0.51$ & $-$ & $0.60$ \\
 			& mFITEXY $(Y|X)$	     & $8.18 \pm 0.06$ & $3.27 \pm 0.21$ & $0.51$ & $0.22^{+0.10}_{-0.07}$ & $0.45$ \\
 			& {\tt linmix\_err} $(Y|X)$  & $8.17 \pm 0.06$ & $3.17 \pm 0.24$ & $0.51$ & $0.29 \pm 0.07$ & $0.56$ \\ [0.5em]
 			& BCES $(X|Y)$  	     & $8.14 \pm 0.08$ & $3.56 \pm 0.25$ & $0.51$ & $-$ & $0.60$ \\
 			& mFITEXY $(X|Y)$	     & $8.18 \pm 0.06$ & $3.51 \pm 0.23$ & $0.51$ & $0.23^{+0.10}_{-0.07}$ & $0.60$ \\
 			& {\tt linmix\_err} $(X|Y)$  & $8.17 \pm 0.07$ & $3.49 \pm 0.26$ & $0.51$ & $0.30 \pm 0.07$ & $0.60$ \\ [0.5em]
 			& BCES Bisector 	     & $8.14 \pm 0.08$ & $3.56 \pm 0.29$ & $0.51$ & $-$ & $0.60$ \\
 			& mFITEXY Bisector	     & $8.18 \pm 0.06$ & $3.39 \pm 0.15$ & $0.51$ & $-$ & $0.58$ \\
 			& {\tt linmix\_err} Bisector & $8.17 \pm 0.07$ & $3.33 \pm 0.18$ & $0.51$ & $-$ & $0.57$ \\ [0.5em]

 Elliptical (E) (30)	& BCES $(Y|X)$  	     & $8.80 \pm 0.53$ & $-18.16 \pm 53.99$ & $0.76$ & $-$ & $3.02$ \\
 			& mFITEXY $(Y|X)$	     & $8.90 \pm 0.10$ & $4.47 \pm 0.88$ & $0.76$ & $0.29^{+0.14}_{-0.10}$ & $0.56$ \\
 			& {\tt linmix\_err} $(Y|X)$  & $8.84 \pm 0.12$ & $3.56 \pm 1.35$ & $0.76$ & $0.44 \pm 0.12$ & $0.59$ \\ [0.5em]
 			& BCES $(X|Y)$  	     & $8.80 \pm 0.18$ & $8.00 \pm 2.55$ & $0.76$ & $-$ & $1.01$ \\
 			& mFITEXY $(X|Y)$	     & $8.92 \pm 0.15$ & $6.85 \pm 1.75$ & $0.76$ & $0.36^{+0.20}_{-0.15}$ & $0.89$ \\
 			& {\tt linmix\_err} $(X|Y)$  & $8.89 \pm 0.20$ & $6.96 \pm 2.49$ & $0.76$ & $0.63 \pm 0.30$ & $0.89$ \\ [0.5em]
 			& BCES Bisector 	     & $8.80 \pm 0.11$ & $-0.03 \pm 0.10$ & $0.76$ & $-$ & $0.64$ \\
 			& mFITEXY Bisector	     & $8.91 \pm 0.13$ & $5.42 \pm 0.85$ & $0.76$ & $-$ & $0.73$ \\
 			& {\tt linmix\_err} Bisector & $8.85 \pm 0.16$ & $4.73 \pm 1.30$ & $0.76$ & $-$ & $0.67$ \\ [0.5em]

 Lenticular (S0) (11)	& BCES $(Y|X)$  	     & $7.75 \pm 0.58$ & $-11.51 \pm 31.78$ & $0.33$ & $-$ & $2.11$ \\
 			& mFITEXY $(Y|X)$	     & $7.65 \pm 0.12$ & $3.78 \pm 1.20$ & $0.33$ & $0.00^{+0.00}_{-0.00}$ & $0.26$ \\
 			& {\tt linmix\_err} $(Y|X)$  & $$ & $$ & $0.33$ & $$ & $$ \\ [0.5em]
 			& BCES $(X|Y)$  	     & $7.75 \pm 0.13$ & $3.54 \pm 0.99$ & $0.33$ & $-$ & $0.46$ \\
 			& mFITEXY $(X|Y)$	     & $7.65 \pm 0.12$ & $3.78 \pm 1.20$ & $0.33$ & $0.00^{+0.00}_{-0.00}$ & $0.49$ \\
 			& {\tt linmix\_err} $(X|Y)$  & $$ & $$ & $0.33$ & $$ & $$ \\ [0.5em]
 			& BCES Bisector 	     & $7.75 \pm 0.13$ & $-0.09 \pm 0.15$ & $0.33$ & $-$ & $0.48$ \\
 			& mFITEXY Bisector	     & $7.65 \pm 0.12$ & $3.78 \pm 0.85$ & $0.33$ & $-$ & $0.49$ \\
 			& {\tt linmix\_err} Bisector & $$ & $$ & $0.33$ & $-$ & $$ \\ [0.5em]

 Spiral (Sp) (17)	& BCES $(Y|X)$  	     & $7.18 \pm 0.28$ & $6.78 \pm 6.62$ & $0.18$ & $-$ & $1.23$ \\
 			& mFITEXY $(Y|X)$	     & $7.24 \pm 0.13$ & $4.48 \pm 0.90$ & $0.18$ & $0.13^{+0.42}_{-0.13}$ & $0.52$ \\
 			& {\tt linmix\_err} $(Y|X)$  & $7.22 \pm 0.16$ & $3.57 \pm 1.36$ & $0.18$ & $0.39 \pm 0.19$ & $0.70$ \\ [0.5em]
 			& BCES $(X|Y)$  	     & $7.18 \pm 0.23$ & $5.48 \pm 1.93$ & $0.18$ & $-$ & $0.99$ \\
 			& mFITEXY $(X|Y)$	     & $7.24 \pm 0.14$ & $4.62 \pm 0.96$ & $0.18$ & $0.13^{+0.43}_{-0.13}$ & $0.85$ \\
 			& {\tt linmix\_err} $(X|Y)$  & $7.21 \pm 0.21$ & $4.86 \pm 1.64$ & $0.18$ & $0.45 \pm 0.31$ & $0.89$ \\ [0.5em]
 			& BCES Bisector 	     & $7.18 \pm 0.25$ & $6.06 \pm 3.66$ & $0.18$ & $-$ & $1.10$ \\
 			& mFITEXY Bisector	     & $7.24 \pm 0.14$ & $4.55 \pm 0.66$ & $0.18$ & $-$ & $0.84$ \\
 			& {\tt linmix\_err} Bisector & $7.22 \pm 0.19$ & $4.12 \pm 1.07$ & $0.18$ & $-$ & $0.77$ \\ [0.5em]

 Early-type (E+S0) (43)	& BCES $(Y|X)$  	     & $8.54 \pm 0.10$ & $4.07 \pm 0.87$ & $0.64$ & $-$ & $0.65$ \\
 			& mFITEXY $(Y|X)$	     & $8.58 \pm 0.07$ & $3.32 \pm 0.34$ & $0.64$ & $0.24^{+0.10}_{-0.07}$ & $0.45$ \\
 			& {\tt linmix\_err} $(Y|X)$  & $8.57 \pm 0.08$ & $3.12 \pm 0.43$ & $0.64$ & $0.32 \pm 0.08$ & $0.53$ \\ [0.5em]
 			& BCES $(X|Y)$  	     & $8.54 \pm 0.09$ & $3.95 \pm 0.55$ & $0.64$ & $-$ & $0.63$ \\
 			& mFITEXY $(X|Y)$	     & $8.59 \pm 0.08$ & $3.88 \pm 0.43$ & $0.64$ & $0.26^{+0.11}_{-0.08}$ & $0.62$ \\
 			& {\tt linmix\_err} $(X|Y)$  & $8.59 \pm 0.09$ & $3.82 \pm 0.50$ & $0.64$ & $0.35 \pm 0.10$ & $0.61$ \\ [0.5em]
 			& BCES Bisector 	     & $8.54 \pm 0.10$ & $4.01 \pm 0.63$ & $0.64$ & $-$ & $0.64$ \\
 			& mFITEXY Bisector	     & $8.59 \pm 0.07$ & $3.58 \pm 0.27$ & $0.64$ & $-$ & $0.58$ \\
 			& {\tt linmix\_err} Bisector & $8.58 \pm 0.08$ & $3.44 \pm 0.33$ & $0.64$ & $-$ & $0.56$ \\ [0.5em]

\tableline 
\tableline
\end{tabular}
\label{tab:lregMn} 
\tablecomments{For each subsample, we indicate $\langle \log n_{\rm sph} \rangle$, its average value of spheroid S\'ersic index. 
In the last two columns, we report $\epsilon$, the intrinsic scatter, and $\Delta$, the total rms scatter in the $L_{\rm sph}$ direction. 
all - mergers - outliers
Both the early- and late-type subsamples do not contain the two galaxies classified as S0/Sp and the two galaxies classified as mergers (45+17=66-2-2).  }
\end{table*}


%%%%%%%%%%%%%%%%%%%%%%%%%%%%%%%%%%%%%%%%%%%%%%%%%%%%%%%%%%%%%%%%%%%%%%%%%%%%%%%%%%%%%%%%%%%%%%%%%%%%%%%%%%%%%%%%%%%%%%%%
%%%%%%%%%%%%%%%%%%%%%%%%%%%%%%%%%%%%%%%%%%%%%%%%%%%%%%%%%%%%%%%%%%%%%%%%%%%%%%%%%%%%%%%%%%%%%%%%%%%%%%%%%%%%%%%%%%%%%%%%



%%%%%%%%%%%%


%%
\acknowledgments
kkkkk


\bibliography{/Users/gsavorgnan/galaxy_vivisection/papers/SMBHbibliography}


\clearpage


\end{document}

