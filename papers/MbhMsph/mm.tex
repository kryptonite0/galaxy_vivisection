
\documentclass[preprint2]{emulateapj}

\usepackage{natbib}
\bibliographystyle{apj}
\usepackage{longtable}
\usepackage[]{graphicx}
\usepackage{amsmath}
\usepackage{natbib}
\usepackage{tabularx}
\usepackage{bm}

%% Sometimes a paper's abstract is too long to fit on the
%% title page in preprint2 mode. When that is the case,
%% use the longabstract style option.

%% \documentclass[preprint2,longabstract]{aastex}

\newcommand{\vdag}{(v)^\dagger}
\newcommand{\myemail}{gsavorgn@astro.swin.edu.au}
\newcommand{\fitfigurewidth}{0.8\textwidth}


\shorttitle{M-M paper}
\shortauthors{Savorgnan et al.}

\begin{document}

\title{M-M paper, morphological dependence, red sequence }

\author{G. A. D. Savorgnan\altaffilmark{1} and A. W. Graham\altaffilmark{1} and A. Marconi and E. Sani and L.K. Hunt}
\affil{Centre for Astrophysics and Supercomputing, Swinburne University of Technology, Hawthorn, Victoria 3122, Australia.}
\email{gsavorgn@astro.swin.edu.au}

%\and

%\author{A. W. Graham\altaffilmark{1}}
%\affil{Centre for Astrophysics and Supercomputing, Swinburne University of Technology, Hawthorn, Victoria 3122, Australia.}

%% Notice that each of these authors has alternate affiliations, which
%% are identified by the \altaffilmark after each name.  Specify alternate
%% affiliation information with \altaffiltext, with one command per each
%% affiliation.

%\altaffiltext{1}{Visiting Astronomer, Cerro Tololo Inter-American Observatory.
%CTIO is operated by AURA, Inc.\ under contract to the National Science
%Foundation.}
%\altaffiltext{2}{Society of Fellows, Harvard University.}
%\altaffiltext{3}{present address: Center for Astrophysics,
%    60 Garden Street, Cambridge, MA 02138}
%\altaffiltext{4}{Visiting Programmer, Space Telescope Science Institute}
%\altaffiltext{5}{Patron, Alonso's Bar and Grill}

%% Mark off your abstract in the ``abstract'' environment. In the manuscript
%% style, abstract will output a Received/Accepted line after the
%% title and affiliation information. No date will appear since the author
%% does not have this information. The dates will be filled in by the
%% editorial office after submission.

\begin{abstract}
We present the latest scaling relations between black hole mass, $M_{\rm BH}$, 
and spheroid luminosity, $L_{\rm sph}$, spheroid stellar mass, $M_{\rm *,sph}$, and galaxy luminosity, $L_{\rm gal}$,
built upon an unprecedented high-quality data set (provided here in a ready-to-use table).
Our sample, the largest ever used, counts 66 local galaxies with a dynamical measurement of $M_{\rm BH}$.
Spheroid luminosities come from high signal-to-noise $3.6\rm~\mu m$ \emph{Spitzer} satellite imagery, 
which is minimally affected by dust and star formation. 
These luminosities were derived from our state-of-the-art multicomponent galaxy decompositions, 
that take into account bulges, disks, bars, spiral arms, rings, haloes, extended or unresolved nuclear sources and partially depleted cores, 
and that -- for the first time -- were checked to be consistent with the galaxy kinematics. 
Previous studies used galaxy samples that were overwhelmingly dominated by high-mass, early-type objects.
Instead, our sample includes 17 spiral galaxies, half of which have $M_{\rm BH} < 10^7\rm~M_\odot$, 
and allows us to investigate the poorly studied low-mass end of the correlations.
We provide updated linear regressions 
(symmetrical, to be compared with theoretical expectations, and non-symmetrical, to be used to predict black hole masses) 
for S\'ersic/core-S\'ersic galaxies and for different morphological types (E/S0, Sp). 
The bulges of early-type (E/S0) galaxies follow $M_{\rm BH} \propto M_{\rm *,sph}^{1.0 \pm 0.1}$, 
as expected from a dry-merging formation scenario,
and define a tight \emph{red sequence} with intrinsic scatter {\bf ???}.
On the other hand, the bulges of late-type (Sp) galaxies define a much steeper \emph{blue sequence}, 
with $M_{\rm BH} \propto M_{\rm *,sph}^{3.0 \pm 1.3}$, 
indicating that gas-rich processes feed the black hole three times more efficiently than the host bulge. 
Disproving some previous claims, we also conclude that: 1) S\'ersic and core-S\'ersic galaxies do not define two distinct sequences; 
2) bulges with S\'ersic index $n_{\rm sph}<2$, argued by some to be pseudo-bulges, 
are not offset to lower $M_{\rm BH}$ from the correlation defined by bulges with $n_{\rm sph}>2$; 
3) $L_{\rm sph}$ and $L_{\rm gal}$ correlate equally well with $M_{\rm BH}$ only for early-type galaxies.



\end{abstract}

\keywords{keywords}

\section{Introduction}
\label{sec:int}
More than two and a half decades ago, 
\cite{dressler1989} foresaw a ``rough scaling of black hole mass with the mass of the spheroidal component'', 
as suggested by the sequence of five galaxies (M87, M104, M31, M32 and the Milky Way). 
His ``rough scaling'' was a premature version of the nowadays popular correlation between black hole mass, $M_{\rm BH}$,  
and host spheroid luminosity, $L_{\rm sph}$, and also host spheroid mass, $M_{\rm sph}$ 
\citep{yee1992,kormendyrichstone1995,magorrian1998,marconihunt2003,haringrix2004}. 
These early studies were dominated by high-mass, early-type galaxies, 
for which they reported a quasi-linear $M_{\rm BH} - M_{\rm sph}$ relation, 
consistent with a dry-merging formation scenario. 
Subsequent studies of the $M_{\rm BH} - L_{\rm sph}$ and $M_{\rm BH} - M_{\rm sph}$ diagrams 
(\citealt{ferrareseford2005,lauer2007,graham2007,graham2008bar,gultelkin2009,sani2011,beifiori2012,erwingadotti2012,
vika2012,vandenbosch2012,mcconnellma2013,kormendyho2013,rusli2013}; 
see \citealt{graham2015bulges} for an extensive review about the early discovery and successive improvements of these correlations)
used similar galaxy samples, which remained dominated by high-mass, early-type objects having $M_{\rm BH} \gtrsim 0.5 \times 10^8~\rm M_\odot$, 
and recovered a near-linear relation. 
However, the consensus about a linear $M_{\rm BH} - M_{\rm sph}$ correlation was not unanimous. 
Some studies reported a slope steeper than one,  
or noticed that low-mass spheroids were downwards offset from the relation traced by their high-mass counterparts 
\citep{laor1998,wandel1999,laor2001,ryan2007}.
Recently, \cite{lasker2014data,lasker2014anal} derived $2.2~\rm \mu m$ bulge luminosities for 35 galaxies 
(among which only 4 were classified as spiral galaxies), 
and reported a slope below unity for their $M_{\rm BH} - M_{\rm sph}$ relation. 
They also claimed that the black hole mass correlates equally well with the total galaxy luminosity 
as it does with the bulge luminosity. \\
The $M_{\rm BH} - L_{\rm sph}$ relation can be predicted from other two correlations involving the bulge velocity dispersion, $\sigma$.
The first of these two is the $M_{\rm BH} - \sigma$ relation \citep{ferraresemerritt2000,gebhardt2000},
which can be described by a single power-law ($M_{\rm BH} \propto \sigma^5$) 
over the range in velocity dispersion $70-350~\rm km~s^{-1}$ (e.g.~\citealt{graham2011,mcconnell2011,grahamscott2013}).
The second is the $L_{\rm sph} - \sigma$ relation, 
which has long been known to be a ``double power-law'', 
being $L_{\rm sph} \propto \sigma^5$ at the luminous end \citep{schechter1980,malumuthkrishner1981,vonderlinden2007,liu2008}, 
and $L_{\rm sph} \propto \sigma^2$ at intermediate and faint luminosities 
\citep{davies1983,held1992,matkovicguzman2005,derijcke2005,balcells2007screl,chilingarian2008,forbes2008,cody2009,tortora2009,kourkchi2012}. 
The change in slope of the $L_{\rm sph} - \sigma$ relation occurs at $M_B \approx -20.5\rm~mag$, 
corresponding to $\sigma \approx 200~\rm km~s^{-1}$. 
That is, the $M_{\rm BH} - L_{\rm sph}$ relation should be better described by a ``broken'', rather than a single, power-law, 
having $M_{\rm BH} \propto L_{\rm sph}^{2.5}$ at the low-luminosity end, 
and $M_{\rm BH} \propto L_{\rm sph}^1$ at the high-luminosity end.  
Due to the scatter in the $M_{\rm BH} - L_{\rm sph}$ (or $M_{\rm BH} - M_{\rm sph}$) diagram, 
studies that have not sufficiently probed below $M_{\rm BH} \approx 10^7\rm~M_\odot$ 
can easily miss the change in slope occuring at $M_{\rm BH} \approx 10^{(8 \pm 1)}\rm~M_\odot$, 
and erroneously recover a single log-linear relation. \\
When \cite{graham2012bent} pointed out this overlooked inconsistency, 
he identified two different populations of galaxies, 
namely the core-S\'ersic \citep{graham2003coresersicmodel,trujillo2004coresersicmodel} and S\'ersic 
spheroids\footnote{Core-S\'ersic spheroids have partially depleted cores relative to their outer S\'ersic light profile, 
whereas S\'ersic spheroids have no central deficit of stars. 
While core-S\'ersic spheroids are also ``core galaxies'', as given by the Nuker definition \citep{lauer2007lumell},
it should be noted that $\sim$20\% of ``core galaxies'' are not core-S\'ersic spheroids 
(\citealt{dullograham2014cores}, their Appendix A.2), i.e. do not have depleted cores.
The change in slope of the $L_{\rm sph} - \sigma$ relation corresponds to the division between 
core-S\'ersic and S\'ersic spheroids (e.g. \citealt{grahamguzman2003}).},
and attributed the change in slope (from log-quadratic to log-linear) to their different formation mechanisms. 
In this scenario, core-S\'ersic spheroids are built in additive dry merger events, 
where the black hole and the bulge grow at the same pace, increasing their mass in lock steps ($M_{\rm BH} \propto L_{\rm sph}^1$), 
whereas S\'ersic spheroids originate from gas-rich processes, 
in which the mass of the black hole increases more rapidly than the mass of its host spheroid ($M_{\rm BH} \propto L_{\rm sph}^{2.5}$). 
\citeauthor{grahamscott2013} (\citeyear{grahamscott2013}, hereafter GS13) and \citeauthor{scott2013} (\citeyear{scott2013}, hereafter S+13) 
presented double power-law linear regressions 
for S\'ersic/core-S\'ersic spheroids in the $M_{\rm BH} - L_{\rm sph}$ and $M_{\rm BH} - M_{\rm *,sph}$ 
(spheroid stellar mass) diagrams, respectively, probing down to $M_{\rm BH} \approx 10^6\rm~M_\odot$. 
To obtain their dust-corrected \emph{bulge} magnitudes, they did not perform bulge/disc decompositions, 
but instead they converted $B-$band and $K_S-$band observed, total \emph{galaxy} magnitudes 
using a mean statistical correction based on each object's morphological type and disc 
inclination\footnote{While this resulted in individual bulge magnitudes not being exactly correct, 
their large sample size allowed them to obtain a reasonably ensemble average correction.}. 
However, this mean statistical correction was obtained from the results of non-modern bulge/disk decompositions, 
which did not include extra components. 
It should also be noted that $\sim$80\% of their core-S\'ersic spheroids were morphologically classified as elliptical galaxies, 
and $\sim$80\% of their S\'ersic spheroids were morphologically classified as bulges of disk galaxies (lenticulars and spirals). \\
Several recent papers \citep{jiang2011a,jiang2013,mathur2012,reines2013} claimed an offset at the low-mass end of the $M_{\rm BH} - M_{\rm *,sph}$ diagram,
such that the black hole mass is lower than expected from the near-linear correlation traced by the high-mass, early-type spheroids. 
However, \cite{grahamscott2015} showed that the low-mass spheroids ($10^{8.5} \lesssim M_{\rm *,sph}/{\rm M_\odot} \lesssim 10^{10.5}$) 
are not somehow randomly offset from the high-mass, near-linear correlation, 
but lie on the two times steeper relation traced by the S\'ersic spheroids. 
{\bf AL: would you like to add a sentence here, mentioning your forthcoming paper with ewan?} \\
Here we investigate substructure in the $M_{\rm BH} - L_{\rm sph}$ and $M_{\rm BH} - M_{\rm *,sph}$ diagrams 
using state-of-the-art galaxy decompositions (Savorgnan \& Graham \emph{in preparation}, hereafter \emph{Paper I}) 
for the largest sample of galaxies with directly measured black hole masses.
Our galaxies are large and nearby, which allows us to perform accurate multicomponent decompositions 
(instead of simple bulge/disk decompositions) without incurring in significant parameter degeneracies. 
Our decompositions were obtained from $3.6\rm~\mu m$ \emph{Spitzer} satellite imagery, 
which is an excellent proxy for the stellar mass, superior to the $K-$band (\citealt{sheth2010} and references therein).
Nine of our galaxies have $M_{\rm BH} \lesssim 10^7\rm~M_\odot$, 
which allows us to accurately constrain the slope of the correlation at the low-mass end.
In addition to this, our galaxy sample includes 17 spiral galaxies, 
representing a notable improvement over the past studies dominated by early-type systems. 
In a forthcoming paper, we will explore the relation between black hole mass and bulge dynamical mass, 
$M_{\rm dyn,sph} \propto R_{\rm e} \sigma^2$, and address the issue of a black hole fundamental plane.
This paper is structured as follows... 

\section{Data}
\label{sec:data}
Our galaxy sample (see Table \ref{tab:sample}) 
consists of 66 objects for which a dynamical measurement of the black hole mass had been reported in the literature 
(GS13, \citealt{rusli2013bhmassesDM}) at the time we started this project, 
and for which we were able to obtain useful bulge parameters from $3.6\rm~\mu m$ \emph{Spitzer} satellite imagery. 
Bulge magnitudes were derived from our state-of-the-art galaxy decompositions, which take into account 
bulges, disks, spiral arms, bars, rings, haloes, extended or unresolved nuclear sources and partially depleted cores.
Kinematical information \citep{atlas3dIII-MNRAS,scott2014,arnold2014} was used 
to confirm the presence of rotationally supported components in most early-type galaxies, 
and to identify their extent (intermediate-scale, embedded disks or large-scale disks). 
\emph{Paper I} will present the dataset used here to investigate the $M_{\rm BH} - L_{\rm sph}$ and $M_{\rm BH} - M_{\rm *,sph}$ diagrams, 
give details about the data reduction process and the sophisticated galaxy modelling technique that we developed, 
discuss how we estimated the uncertainties on the bulge parameters, 
and illustrate the individual 66 galaxy decompositions. \\
Bulge luminosities\footnote{Absolute luminosities were calculated 
assuming a $3.6\rm~\mu m$ solar absolute magnitude of $3.25\rm~mag$ \citep{sani2011}.} 
were first converted into stellar masses using a constant $3.6\rm~\mu m$ mass-to-light ratio, $\Gamma_{3.6} = 0.6$ \citep{meidt2014}.
We then explored a more sophisticated way to compute mass-to-light ratios, 
using the color-$\Gamma_{3.6}$ relation published by 
\citeauthor{meidt2014} (\citeyear{meidt2014}, their equation 4), 
which allows one to estimate $\Gamma_{3.6}$ of a galaxy from its $[3.6] - [4.5]$ color. 
Individual $[3.6] - [4.5]$ colors\footnote{These are integrated $[3.6] - [4.5]$ colors, measured in a circular aperture 
within one galaxy's effective radius.} were taken from 
\citeauthor{peletier2012} (\citeyear{peletier2012}, column 8 of their Table 1) 
when available for our galaxies, 
or were estimated from the bulge stellar velocity dispersion, $\sigma$, 
using the color-$\sigma$ relation presented by \citeauthor{peletier2012} (\citeyear{peletier2012}, their Figure 6).
We found that the range in $[3.6] - [4.5]$ color is small ($0.06\rm~mag$), 
and thus the range in $\Gamma_{3.6}$ is also small ($0.04$).
After checking that using a single $\Gamma_{3.6} = 0.6$, independent of $[3.6] - [4.5]$ color, 
does not significantly affect the results of our analysis, 
we decided to use individual, color-dependent mass-to-light ratios. \\
The S\'ersic/core-S\'ersic classification presented in this work 
comes from the compilation of \citet{savorgnangraham2014},
who identified partially depleted cores according to the same criteria used by GS13.
When no high-resolution image analysis was available from the literature, 
they inferred the presence of a partially depleted core based on the stellar velocity dispersion:
a galaxy is classified as core-S\'ersic if $\sigma > 270\rm~km~s^{-1}$,
or as S\'ersic if $\sigma < 166\rm~km~s^{-1}$. \\
For each galaxy, the total luminosity (or galaxy luminosity, $L_{\rm gal}$) is the sum of the luminosities of all its sub-components. 
Due to the complexity of their modelling, 
four galaxies (see Table \ref{tab:sample}, column 7) had their galaxy luminosities 
underestimated\footnote{These four cases will be discussed in \emph{Paper I}.}, 
which are given here as lower limits. 

\begin{table*}                                        
\small                                                
\begin{center}                                        
\caption{Galaxy sample.} 
\begin{tabular}{llllllrll}                           
\tableline                                                
\multicolumn{1}{l}{{\bf Galaxy}} &                   
\multicolumn{1}{l}{{\bf Type}} &                     
\multicolumn{1}{l}{{\bf Core}} &                     
\multicolumn{1}{l}{{\bf Distance}} &                 
\multicolumn{1}{l}{{\bf $\bm{M_{\rm BH}}$}} &  
\multicolumn{1}{l}{{\bf $\bm{MAG_{\rm sph}}$}} &  
\multicolumn{1}{l}{{\bf $\bm{MAG_{\rm gal}}$}} &  
\multicolumn{1}{l}{{\bf $\bm{[3.6]-[4.5]}$}} &  
\multicolumn{1}{l}{{\bf $\bm{M_{\rm *,sph}}$}} \\  
\multicolumn{1}{l}{} &                                
\multicolumn{1}{l}{} &                                
\multicolumn{1}{l}{} &                                
\multicolumn{1}{l}{[Mpc]} &                           
\multicolumn{1}{l}{$[10^8~\rm M_{\odot}]$} &         
\multicolumn{1}{l}{[mag]} &                                
\multicolumn{1}{l}{[mag]} &                                
\multicolumn{1}{l}{[mag]} &                                
\multicolumn{1}{l}{$[10^{10}~\rm M_{\odot}]$} \\                             
\multicolumn{1}{l}{(1)} &                             
\multicolumn{1}{l}{(2)} &                             
\multicolumn{1}{l}{(3)} &                             
\multicolumn{1}{l}{(4)} &                             
\multicolumn{1}{l}{(5)} &                             
\multicolumn{1}{l}{(6)} &                             
\multicolumn{1}{l}{(7)} &                             
\multicolumn{1}{l}{(8)} &                             
\multicolumn{1}{l}{(9)} \\                         
\tableline                                                
IC 1459  &  E  &  yes   &  $28.4$  &  $24_{-10}^{+10}$   &  $-26.15_{-0.11}^{+0.18}$   &  $-26.15 \pm 0.25$ 
 &  $-0.12$  &  $27_{-23}^{+30}$   \\ 
IC 2560  &  Sp (bar)  &  no?  &  $40.7$  &  $0.044_{-0.022}^{+0.044}$   &  $-22.27_{-0.58}^{+0.66}$   &  $-24.76 \pm 0.25$ 
 &  $-0.08$  &  $1.0_{-0.6}^{+1.8}$   \\ 
IC 4296  &  E  &  yes?  &  $40.7$  &  $11_{-2}^{+2}$   &  $-26.35_{-0.11}^{+0.18}$   &  $-26.35 \pm 0.25$ 
 &  $-0.12$  &  $31_{-26}^{+34}$   \\ 
M104  &  S0/Sp  &  yes   &  $9.5$  &  $6.4_{-0.4}^{+0.4}$   &  $-23.91_{-0.58}^{+0.66}$   &  $-25.21 \pm 0.25$ 
 &  $-0.12$  &  $3.4_{-1.9}^{+5.8}$   \\ 
M105  &  E  &  yes   &  $10.3$  &  $4_{-1}^{+1}$   &  $-24.29_{-0.58}^{+0.66}$   &  $-24.29 \pm 0.25$ 
 &  $-0.10$  &  $5.6_{-3.0}^{+9.5}$   \\ 
M106  &  Sp (bar)  &  no   &  $7.2$  &  $0.39_{-0.01}^{+0.01}$   &  $-21.11_{-0.11}^{+0.18}$   &  $-24.04 \pm 0.25$ 
 &  $-0.08$  &  $0.37_{-0.31}^{+0.41}$   \\ 
M31  &  Sp (bar)  &  no   &  $0.7$  &  $1.4_{-0.3}^{+0.9}$   &  $-22.74_{-0.11}^{+0.18}$   &  $-24.67 \pm 0.25$ 
 &  $-0.09$  &  $1.5_{-1.3}^{+1.6}$   \\ 
M49  &  E  &  yes   &  $17.1$  &  $25_{-1}^{+3}$   &  $-26.54_{-0.11}^{+0.18}$   &  $-26.54 \pm 0.25$ 
 &  $-0.12$  &  $39_{-33}^{+43}$   \\ 
M59  &  E  &  no   &  $17.8$  &  $3.9_{-0.4}^{+0.4}$   &  $-25.18_{-0.11}^{+0.18}$   &  $-25.27 \pm 0.25$ 
 &  $-0.09$  &  $14_{-11}^{+15}$   \\ 
M64  &  Sp  &  no?  &  $7.3$  &  $0.016_{-0.004}^{+0.004}$   &  $-21.54_{-0.11}^{+0.18}$   &  $-24.24 \pm 0.25$ 
 &  $-0.06$  &  $0.64_{-0.55}^{+0.71}$   \\ 
M81  &  Sp (bar)  &  no   &  $3.8$  &  $0.74_{-0.11}^{+0.21}$   &  $-23.01_{-0.66}^{+0.88}$   &  $-24.43 \pm 0.25$ 
 &  $-0.09$  &  $1.9_{-0.9}^{+3.6}$   \\ 
M84  &  E  &  yes   &  $17.9$  &  $9.0_{-0.8}^{+0.9}$   &  $-26.01_{-0.58}^{+0.66}$   &  $-26.01 \pm 0.25$ 
 &  $-0.10$  &  $28_{-15}^{+47}$   \\ 
M87  &  E  &  yes   &  $15.6$  &  $58.0_{-3.5}^{+3.5}$   &  $-26.00_{-0.58}^{+0.66}$   &  $-26.00 \pm 0.25$ 
 &  $-0.11$  &  $26_{-14}^{+44}$   \\ 
M89  &  E  &  yes   &  $14.9$  &  $4.7_{-0.5}^{+0.5}$   &  $-24.48_{-0.58}^{+0.66}$   &  $-24.74 \pm 0.25$ 
 &  $-0.11$  &  $6.3_{-3.4}^{+10.7}$   \\ 
M94  &  Sp (bar)  &  no?  &  $4.4$  &  $0.060_{-0.014}^{+0.014}$   &  $-22.08_{-0.11}^{+0.18}$   &  $\leq-23.36$   &  $-0.07$  &  $1.00_{-0.85}^{+1.11}$   \\ 
M96  &  Sp (bar)  &  no   &  $10.1$  &  $0.073_{-0.015}^{+0.015}$   &  $-22.15_{-0.11}^{+0.18}$   &  $-24.20 \pm 0.25$ 
 &  $-0.08$  &  $0.97_{-0.83}^{+1.08}$   \\ 
NGC 0524  &  S0  &  yes   &  $23.3$  &  $8.3_{-1.3}^{+2.7}$   &  $-23.19_{-0.11}^{+0.18}$   &  $-24.92 \pm 0.25$ 
 &  $-0.09$  &  $2.2_{-1.9}^{+2.5}$   \\ 
NGC 0821  &  E  &  no   &  $23.4$  &  $0.39_{-0.09}^{+0.26}$   &  $-24.00_{-0.66}^{+0.88}$   &  $-24.26 \pm 0.25$ 
 &  $-0.09$  &  $4.7_{-2.1}^{+8.7}$   \\ 
NGC 1023  &  S0 (bar)  &  no   &  $11.1$  &  $0.42_{-0.04}^{+0.04}$   &  $-22.82_{-0.11}^{+0.18}$   &  $-24.20 \pm 0.25$ 
 &  $-0.10$  &  $1.5_{-1.3}^{+1.7}$   \\ 
NGC 1300  &  Sp (bar)  &  no   &  $20.7$  &  $0.73_{-0.35}^{+0.69}$   &  $-22.06_{-0.58}^{+0.66}$   &  $-24.16 \pm 0.25$ 
 &  $-0.10$  &  $0.70_{-0.38}^{+1.19}$   \\ 
NGC 1316  &  merger  &  no   &  $18.6$  &  $1.50_{-0.80}^{+0.75}$   &  $-24.89_{-0.58}^{+0.66}$   &  $-26.48 \pm 0.25$ 
 &  $-0.10$  &  $9.5_{-5.2}^{+16.2}$   \\ 
NGC 1332  &  E/S0  &  no   &  $22.3$  &  $14_{-2}^{+2}$   &  $-24.89_{-0.66}^{+0.88}$   &  $-24.95 \pm 0.25$ 
 &  $-0.12$  &  $8.2_{-3.6}^{+15.0}$   \\ 
NGC 1374  &  E  &  no?  &  $19.2$  &  $5.8_{-0.5}^{+0.5}$   &  $-23.68_{-0.11}^{+0.18}$   &  $-23.70 \pm 0.25$ 
 &  $-0.09$  &  $3.6_{-3.0}^{+4.0}$   \\ 
NGC 1399  &  E  &  yes   &  $19.4$  &  $4.7_{-0.6}^{+0.6}$   &  $-26.43_{-0.11}^{+0.18}$   &  $-26.46 \pm 0.25$ 
 &  $-0.12$  &  $33_{-28}^{+37}$   \\ 
NGC 2273  &  Sp (bar)  &  no   &  $28.5$  &  $0.083_{-0.004}^{+0.004}$   &  $-23.00_{-0.58}^{+0.66}$   &  $-24.21 \pm 0.25$ 
 &  $-0.08$  &  $2.0_{-1.1}^{+3.4}$   \\ 
NGC 2549  &  S0 (bar)  &  no   &  $12.3$  &  $0.14_{-0.13}^{+0.02}$   &  $-21.25_{-0.11}^{+0.18}$   &  $-22.60 \pm 0.25$ 
 &  $-0.10$  &  $0.35_{-0.30}^{+0.39}$   \\ 
NGC 2778  &  S0 (bar)  &  no   &  $22.3$  &  $0.15_{-0.10}^{+0.09}$   &  $-20.80_{-0.58}^{+0.66}$   &  $-22.44 \pm 0.25$ 
 &  $-0.09$  &  $0.25_{-0.14}^{+0.43}$   \\ 
NGC 2787  &  S0 (bar)  &  no   &  $7.3$  &  $0.40_{-0.05}^{+0.04}$   &  $-20.11_{-0.58}^{+0.66}$   &  $-22.28 \pm 0.25$ 
 &  $-0.10$  &  $0.12_{-0.07}^{+0.20}$   \\ 
NGC 2974  &  Sp (bar)  &  no   &  $20.9$  &  $1.7_{-0.2}^{+0.2}$   &  $-22.95_{-0.58}^{+0.66}$   &  $-24.16 \pm 0.25$ 
 &  $-0.09$  &  $1.8_{-1.0}^{+3.1}$   \\ 
NGC 3079  &  Sp (bar)  &  no?  &  $20.7$  &  $0.024_{-0.012}^{+0.024}$   &  $-23.01_{-0.58}^{+0.66}$   &  $\leq-24.45$   &  $-0.07$  &  $2.4_{-1.3}^{+4.0}$   \\ 
NGC 3091  &  E  &  yes   &  $51.2$  &  $36_{-2}^{+1}$   &  $-26.28_{-0.11}^{+0.18}$   &  $-26.28 \pm 0.25$ 
 &  $-0.12$  &  $30_{-26}^{+34}$   \\ 
NGC 3115  &  E/S0  &  no   &  $9.4$  &  $8.8_{-2.7}^{+10.0}$   &  $-24.22_{-0.11}^{+0.18}$   &  $-24.40 \pm 0.25$ 
 &  $-0.11$  &  $4.9_{-4.1}^{+5.4}$   \\ 
NGC 3227  &  Sp (bar)  &  no   &  $20.3$  &  $0.14_{-0.06}^{+0.10}$   &  $-21.76_{-0.58}^{+0.66}$   &  $-24.26 \pm 0.25$ 
 &  $-0.08$  &  $0.67_{-0.37}^{+1.15}$   \\ 
NGC 3245  &  S0 (bar)  &  no   &  $20.3$  &  $2.0_{-0.5}^{+0.5}$   &  $-22.43_{-0.11}^{+0.18}$   &  $-23.88 \pm 0.25$ 
 &  $-0.10$  &  $1.0_{-0.9}^{+1.1}$   \\ 
NGC 3377  &  E  &  no   &  $10.9$  &  $0.77_{-0.06}^{+0.04}$   &  $-23.49_{-0.58}^{+0.66}$   &  $-23.57 \pm 0.25$ 
 &  $-0.06$  &  $4.0_{-2.2}^{+6.8}$   \\ 
NGC 3384  &  S0 (bar)  &  no   &  $11.3$  &  $0.17_{-0.02}^{+0.01}$   &  $-22.43_{-0.11}^{+0.18}$   &  $-23.74 \pm 0.25$ 
 &  $-0.08$  &  $1.2_{-1.0}^{+1.3}$   \\ 
NGC 3393  &  Sp (bar)  &  no   &  $55.2$  &  $0.34_{-0.02}^{+0.02}$   &  $-23.48_{-0.58}^{+0.66}$   &  $-25.29 \pm 0.25$ 
 &  $-0.10$  &  $2.8_{-1.5}^{+4.7}$   \\ 
NGC 3414  &  E  &  no   &  $24.5$  &  $2.4_{-0.3}^{+0.3}$   &  $-24.35_{-0.11}^{+0.18}$   &  $-24.42 \pm 0.25$ 
 &  $-0.09$  &  $6.5_{-5.5}^{+7.2}$   \\ 
NGC 3489  &  S0/Sp (bar)  &  no   &  $11.7$  &  $0.058_{-0.008}^{+0.008}$   &  $-21.13_{-0.58}^{+0.66}$   &  $-23.07 \pm 0.25$ 
 &  $-0.06$  &  $0.42_{-0.23}^{+0.72}$   \\ 
NGC 3585  &  E  &  no   &  $19.5$  &  $3.1_{-0.6}^{+1.4}$   &  $-25.52_{-0.58}^{+0.66}$   &  $-25.55 \pm 0.25$ 
 &  $-0.10$  &  $18_{-10}^{+30}$   \\ 
NGC 3607  &  E  &  no   &  $22.2$  &  $1.3_{-0.5}^{+0.5}$   &  $-25.36_{-0.58}^{+0.66}$   &  $-25.45 \pm 0.25$ 
 &  $-0.10$  &  $15_{-8}^{+25}$   \\ 
NGC 3608  &  E  &  yes   &  $22.3$  &  $2.0_{-0.6}^{+1.1}$   &  $-24.50_{-0.58}^{+0.66}$   &  $-24.50 \pm 0.25$ 
 &  $-0.08$  &  $7.8_{-4.3}^{+13.4}$   \\ 
NGC 3842  &  E  &  yes   &  $98.4$  &  $97_{-26}^{+30}$   &  $-27.00_{-0.11}^{+0.18}$   &  $-27.04 \pm 0.25$ 
 &  $-0.11$  &  $61_{-52}^{+68}$   \\ 
NGC 3998  &  S0 (bar)  &  no   &  $13.7$  &  $8.1_{-1.9}^{+2.0}$   &  $-22.32_{-0.66}^{+0.88}$   &  $-23.53 \pm 0.25$ 
 &  $-0.12$  &  $0.78_{-0.35}^{+1.43}$   \\ 
NGC 4026  &  S0 (bar)  &  no   &  $13.2$  &  $1.8_{-0.3}^{+0.6}$   &  $-21.58_{-0.66}^{+0.88}$   &  $-23.16 \pm 0.25$ 
 &  $-0.09$  &  $0.50_{-0.22}^{+0.92}$   \\ 
NGC 4151  &  Sp (bar)  &  no   &  $20.0$  &  $0.65_{-0.07}^{+0.07}$   &  $-23.40_{-0.58}^{+0.66}$   &  $-24.44 \pm 0.25$ 
 &  $-0.09$  &  $2.8_{-1.5}^{+4.8}$   \\ 
NGC 4261  &  E  &  yes   &  $30.8$  &  $5_{-1}^{+1}$   &  $-25.72_{-0.58}^{+0.66}$   &  $-25.76 \pm 0.25$ 
 &  $-0.12$  &  $18_{-10}^{+30}$   \\ 
NGC 4291  &  E  &  yes   &  $25.5$  &  $3.3_{-2.5}^{+0.9}$   &  $-24.05_{-0.58}^{+0.66}$   &  $-24.05 \pm 0.25$ 
 &  $-0.11$  &  $3.9_{-2.1}^{+6.7}$   \\ 
NGC 4388  &  Sp (bar)  &  no?  &  $17.0$  &  $0.075_{-0.002}^{+0.002}$   &  $-21.26_{-0.66}^{+0.88}$   &  $\leq-23.50$   &  $-0.07$  &  $0.46_{-0.21}^{+0.85}$   \\ 
NGC 4459  &  S0  &  no   &  $15.7$  &  $0.68_{-0.13}^{+0.13}$   &  $-23.48_{-0.58}^{+0.66}$   &  $-24.01 \pm 0.25$ 
 &  $-0.09$  &  $2.9_{-1.6}^{+5.0}$   \\ 
NGC 4473  &  E  &  no   &  $15.3$  &  $1.2_{-0.9}^{+0.4}$   &  $-23.88_{-0.58}^{+0.66}$   &  $-24.11 \pm 0.25$ 
 &  $-0.10$  &  $3.9_{-2.1}^{+6.6}$   \\ 
NGC 4564  &  S0  &  no   &  $14.6$  &  $0.60_{-0.09}^{+0.03}$   &  $-22.30_{-0.11}^{+0.18}$   &  $-22.99 \pm 0.25$ 
 &  $-0.11$  &  $0.82_{-0.70}^{+0.91}$   \\ 
NGC 4596  &  S0 (bar)  &  no   &  $17.0$  &  $0.79_{-0.33}^{+0.38}$   &  $-22.73_{-0.11}^{+0.18}$   &  $-24.18 \pm 0.25$ 
 &  $-0.08$  &  $1.6_{-1.3}^{+1.7}$   \\ 
\tableline         
\end{tabular}   
\label{tab:sample} 
\end{center}    
\end{table*}    

\begin{table*}                                        
\small                                                
\begin{center}                                        
\begin{tabular}{llllllrll}                           
\tableline                                                
\multicolumn{1}{l}{{\bf Galaxy}} &                   
\multicolumn{1}{l}{{\bf Type}} &                     
\multicolumn{1}{l}{{\bf Core}} &                     
\multicolumn{1}{l}{{\bf Distance}} &                 
\multicolumn{1}{l}{{\bf $\bm{M_{\rm BH}}$}} &  
\multicolumn{1}{l}{{\bf $\bm{MAG_{\rm sph}}$}} &  
\multicolumn{1}{l}{{\bf $\bm{MAG_{\rm gal}}$}} &  
\multicolumn{1}{l}{{\bf $\bm{[3.6]-[4.5]}$}} &  
\multicolumn{1}{l}{{\bf $\bm{M_{\rm *,sph}}$}} \\  
\multicolumn{1}{l}{} &                                
\multicolumn{1}{l}{} &                                
\multicolumn{1}{l}{} &                                
\multicolumn{1}{l}{[Mpc]} &                           
\multicolumn{1}{l}{$[10^8~\rm M_{\odot}]$} &         
\multicolumn{1}{l}{[mag]} &                                
\multicolumn{1}{l}{[mag]} &                                
\multicolumn{1}{l}{[mag]} &                                
\multicolumn{1}{l}{$[10^{10}~\rm M_{\odot}]$} \\                             
\multicolumn{1}{l}{(1)} &                             
\multicolumn{1}{l}{(2)} &                             
\multicolumn{1}{l}{(3)} &                             
\multicolumn{1}{l}{(4)} &                             
\multicolumn{1}{l}{(5)} &                             
\multicolumn{1}{l}{(6)} &                             
\multicolumn{1}{l}{(7)} &                             
\multicolumn{1}{l}{(8)} &                             
\multicolumn{1}{l}{(9)} \\                         
\tableline                                                
NGC 4697  &  E  &  no   &  $11.4$  &  $1.8_{-0.1}^{+0.2}$   &  $-24.82_{-0.66}^{+0.88}$   &  $-24.94 \pm 0.25$ 
 &  $-0.09$  &  $10_{-4}^{+18}$   \\ 
NGC 4889  &  E  &  yes   &  $103.2$  &  $210_{-160}^{+160}$   &  $-27.54_{-0.11}^{+0.18}$   &  $-27.54 \pm 0.25$ 
 &  $-0.12$  &  $91_{-77}^{+101}$   \\ 
NGC 4945  &  Sp (bar)  &  no?  &  $3.8$  &  $0.014_{-0.007}^{+0.014}$   &  $-20.96_{-0.58}^{+0.66}$   &  $\leq-23.79$   &  $-0.06$  &  $0.36_{-0.20}^{+0.62}$   \\ 
NGC 5077  &  E  &  yes   &  $41.2$  &  $7.4_{-3.0}^{+4.7}$   &  $-25.45_{-0.11}^{+0.18}$   &  $-25.45 \pm 0.25$ 
 &  $-0.11$  &  $15_{-13}^{+17}$   \\ 
NGC 5128  &  merger  &  no?  &  $3.8$  &  $0.45_{-0.10}^{+0.17}$   &  $-23.89_{-0.66}^{+0.88}$   &  $-24.97 \pm 0.25$ 
 &  $-0.07$  &  $5.0_{-2.2}^{+9.1}$   \\ 
NGC 5576  &  E  &  no   &  $24.8$  &  $1.6_{-0.4}^{+0.3}$   &  $-24.44_{-0.11}^{+0.18}$   &  $-24.44 \pm 0.25$ 
 &  $-0.09$  &  $7.1_{-6.0}^{+7.9}$   \\ 
NGC 5845  &  S0  &  no   &  $25.2$  &  $2.6_{-1.5}^{+0.4}$   &  $-22.96_{-0.66}^{+0.88}$   &  $-23.10 \pm 0.25$ 
 &  $-0.12$  &  $1.4_{-0.6}^{+2.6}$   \\ 
NGC 5846  &  E  &  yes   &  $24.2$  &  $11_{-1}^{+1}$   &  $-25.81_{-0.58}^{+0.66}$   &  $-25.81 \pm 0.25$ 
 &  $-0.10$  &  $22_{-12}^{+38}$   \\ 
NGC 6251  &  E  &  yes?  &  $104.6$  &  $5_{-2}^{+2}$   &  $-26.75_{-0.11}^{+0.18}$   &  $-26.75 \pm 0.25$ 
 &  $-0.12$  &  $46_{-39}^{+51}$   \\ 
NGC 7052  &  E  &  yes   &  $66.4$  &  $3.7_{-1.5}^{+2.6}$   &  $-26.32_{-0.11}^{+0.18}$   &  $-26.32 \pm 0.25$ 
 &  $-0.11$  &  $33_{-28}^{+36}$   \\ 
NGC 7619  &  E  &  yes   &  $51.5$  &  $25_{-3}^{+8}$   &  $-26.35_{-0.58}^{+0.66}$   &  $-26.41 \pm 0.25$ 
 &  $-0.11$  &  $33_{-18}^{+56}$   \\ 
NGC 7768  &  E  &  yes   &  $112.8$  &  $13_{-4}^{+5}$   &  $-26.90_{-0.58}^{+0.66}$   &  $-26.90 \pm 0.25$ 
 &  $-0.11$  &  $57_{-31}^{+98}$   \\ 
UGC 03789  &  Sp (bar)  &  no?  &  $48.4$  &  $0.108_{-0.005}^{+0.005}$   &  $-22.77_{-0.66}^{+0.88}$   &  $-24.20 \pm 0.25$ 
 &  $-0.07$  &  $1.9_{-0.8}^{+3.4}$   \\ 
\tableline         
\end{tabular}   
\tablecomments{\emph{Column (1)}: Galaxy name. 
\emph{Column (2)}: Morphological type (E=elliptical, S0=lenticular, Sp=spiral, merger). 		       The morphological classification of four galaxies is uncertain (E/S0 or S0/Sp). 		       The presence of a bar is indicated. 
\emph{Column (3)}: Presence of a partially depleted core. 			The question mark is used when the classification has come from the velocity dispersion criteria mentioned in Section \ref{sec:data}. 
\emph{Column (4)}: Distance. 
\emph{Column (5)}: Black hole mass. 
\emph{Column (6)}: Absolute $3.6\rm~\mu m$ bulge magnitude. 		       Bulge magnitudes come from our state-of-the-art multicomponent galaxy decompositions (\emph{Paper I}), 		       which include bulges, disks, bars, spiral arms, rings, haloes, extended or unresolved nuclear sources and partially depleted cores,                        and that -- for the first time -- were checked to be consistent with the galaxy kinematics. 		       The uncertainties were estimated with a method that takes into account systematic errors, which are typically not considered by popular 2D fitting codes. 
\emph{Column (7)}: Absolute $3.6\rm~\mu m$ galaxy magnitude. 			Four galaxies had their magnitudes overestimated, which are give here as upper limits. 
\emph{Column (8)}: $[3.6]-[4.5]$ colour. 
\emph{Column (9)}: Bulge stellar mass. } 
\end{center}    
\end{table*}    




\begin{figure}[h]
\begin{center}
\includegraphics[width=\columnwidth]{images/mbh_vs_mag_sph.pdf}
\caption{Black hole mass against $3.6\rm~\mu m$ spheroid absolute magnitude. 
Symbols are coded according to the galaxy morphological type: red circle = elliptical, red star = elliptical/lenticular, 
red upward triangle = lenticular, blue downward triangle = lenticular/spiral, blue square = spiral, black diamond = merger. 
Empty symbols represent core-S\'ersic spheroids, whereas filled symbols are used for S\'ersic spheroids. 
The red dashed line indicates the BCES bisector linear regression for early-type galaxies (ellipticals+lenticulars), 
with the red shaded area denoting its $1\sigma$ uncertainty. 
The blue solid line shows the BCES bisector linear regression for spiral galaxies, 
with the blue shaded area denoting its $1\sigma$ uncertainty. 
The black dashed-dotted and dotted lines represent the BCES bisector linear regressions for core-S\'ersic and S\'ersic spheroids, respectively.}
\label{fig:mbhmagsph}
\end{center}
\end{figure}

\begin{figure}[h]
\begin{center}
\includegraphics[width=\columnwidth]{images/mbh_vs_mag_sph_psb.pdf}
\caption{Black hole mass against $3.6\rm~\mu m$ spheroid absolute magnitude. 
Symbols are color coded according to the spheroid S\'ersic index $n_{\rm sph}$. 
Bulges with $n_{\rm sph}<2$, claimed by some to be pseudobulges, are marked with an empty square. 
The black solid line shows the BCES bisector linear regression for spheroids with $n_{\rm sph} \geq 2$. 
The inset displays, for all spheroids, the vertical offset 
from the correlation defined by spheroids with $n_{\rm sph} \geq 2$ only 
versus the spheroid S\'ersic index. 
The vertical dashed line corresponds to $n_{\rm sph} = 2$.
Bulges with $n_{\rm sph}<2$ are not randomly offset to lower black hole masses 
from the correlation traced by bulges with $n_{\rm sph} \geq 2$.}
\label{fig:pseudob}
\end{center}
\end{figure}

\begin{figure}[h]
\begin{center}
\includegraphics[width=\columnwidth]{images/mbh_vs_mag_tot.pdf}
\caption{Black hole mass against $3.6\rm~\mu m$ galaxy absolute magnitude. 
Symbols are coded as in Figure \ref{fig:mbhmagsph}.
Four spiral galaxies had their magnitudes overestimated and are shown as upper limits. 
}
\label{fig:mbhmaggal}
\end{center}
\end{figure}



\begin{figure}[h]
\begin{center}
%\includegraphics[width=\columnwidth]{images/mbh_vs_mass_sph_agn.pdf}
\includegraphics[width=\columnwidth]{images/mbh_vs_mass_sph_galsymb_agn.pdf}
\caption{Black hole mass against spheroid stellar mass. 
The galaxies from our sample are shown with large symbols (red = E, orange = E/S0 and S0, blue = S0/Sp and Sp, black = merger).
The 139 small blue dots represent the low-mass AGN sample from \cite{jiang2011a}. 
The blue crossed circles denote 35 additional galaxies (see \citealt{grahamscott2015}, their Table 2), 
some of which might have had their bulge stellar masses overestimated.
The red dashed line indicates the BCES bisector linear regression for early-type galaxies (ellipticals+lenticulars), 
with the red shaded area denoting its $1\sigma$ uncertainty. 
The blue solid line shows the BCES bisector linear regression for spiral galaxies, 
with the blue shaded area denoting its $1\sigma$ uncertainty. }
\label{fig:mbhmasssph}
\end{center}
\end{figure}



\begin{table*}
\centering
\caption{Linear regression analysis. {\bf X0 is the average X of the subsample...}}
\begin{tabular}{llccccc}
\hline
\hline
{\bf Subsample (size)} & {\bf Regression} & $\boldsymbol \alpha$ & $\boldsymbol \beta$ & $\boldsymbol X_0$ & $\boldsymbol \epsilon$ & $\boldsymbol \Delta$ \\ 
\hline 
\\
 & \multicolumn{6}{l}{\emph{Black hole mass -- spheroid luminosity}} \\
 & \multicolumn{6}{l}{$\log[M_{\rm BH}/{\rm M_\odot}] = \alpha + \beta[(MAG_{\rm sph} - X_0)/{\rm mag}]$} \\ [0.5em]
All (66)               & BCES OLS$(Y|X)$   & $8.16 \pm 0.07$ & $-0.44 \pm 0.04$ & $-23.86$ & $-$ & $0.56$ \\
                       & BCES OLS$(X|Y)$   & $8.16 \pm 0.08$ & $-0.61 \pm 0.05$ & $-23.86$ & $-$ & $0.68$ \\
                       & BCES Bisector     & $8.16 \pm 0.07$ & $-0.52 \pm 0.04$ & $-23.86$ & $-$ & $0.60$ \\
                       & mFITEXY OLS$(Y|X)$ & $8.17^{+0.07}_{-0.06}$ & $-0.43^{+0.03}_{-0.03}$ & $-23.86$ & $0.49$ & $0.56$ \\
                       & mFITEXY OLS$(X|Y)$ & $8.15^{+0.07}_{-0.07}$ & $-0.61^{+0.04}_{-0.05}$ & $-23.86$ & $0.96$ & $0.67$ \\
                       & mFITEXY Bisector   & $8.16^{+0.07}_{-0.07}$ & $-0.51^{+0.04}_{-0.04}$ & $-23.86$ & $-$    & $0.60$ \\
                       & Kelly OLS$(Y|X)$  & $8.16 \pm 0.07$ & $-0.42 \pm 0.04$ & $-23.86$ & $0.51 \pm 0.06$ & $0.56$ \\
                       & Kelly OLS$(X|Y)$  & $8.16 \pm 0.09$ & $-0.60 \pm 0.06$ & $-23.86$ & $0.6 \pm 0.1$ & $0.67$ \\
                       & Kelly Bisector    & $$ & $$ & $-23.86$ & $-$    & $$ \\

$n>2$ (43)             & BCES OLS$(Y|X)$   & $8.58 \pm 0.07$ & $-0.42 \pm 0.06$ & $-24.77$ & $-$ & $0.46$ \\
                       & BCES OLS$(X|Y)$   & $8.58 \pm 0.08$ & $-0.58 \pm 0.06$ & $-24.77$ & $-$ & $0.56$ \\
                       & BCES Bisector     & $8.58 \pm 0.07$ & $-0.50 \pm 0.05$ & $-24.77$ & $-$ & $0.49$ \\
                       & mFITEXY OLS$(Y|X)$ & $8.57^{+0.07}_{-0.06}$ & $-0.41^{+0.04}_{-0.04}$ & $-24.77$ & $0.38$ & $0.46$ \\
                       & mFITEXY OLS$(X|Y)$ & $8.56^{+0.08}_{-0.08}$ & $-0.57^{+0.06}_{-0.07}$ & $-24.77$ & $0.78$ & $0.55$ \\
                       & mFITEXY Bisector   & $8.57^{+0.07}_{-0.07}$ & $-0.49^{+0.05}_{-0.05}$ & $-24.77$ & $-$    & $0.49$ \\
                       & Kelly OLS$(Y|X)$  & $8.56 \pm 0.07$ & $-0.39 \pm 0.05$ & $-24.77$ & $0.41 \pm 0.06$ & $0.46$ \\
                       & Kelly OLS$(X|Y)$  & $8.55 \pm 0.09$ & $-0.59 \pm 0.08$ & $-24.77$ & $0.5 \pm  0.1$ & $0.56$ \\
                       & Kelly Bisector    & $$ & $$ & $-24.77$ & $-$    & $$ \\
                   
Core-S\'ersic (22) & BCES OLS$(Y|X)$   & $9.06 \pm 0.09$ & $-0.3  \pm 0.1$  & $-25.73$ & $-$    & $0.42$ \\ 
                   & BCES OLS$(X|Y)$   & $9.1  \pm 0.1$  & $-0.6  \pm 0.1$  & $-25.73$ & $-$    & $0.61$ \\
                   & BCES Bisector     & $9.1  \pm 0.1$  & $-0.47 \pm 0.08$ & $-25.73$ & $-$    & $0.48$ \\
                   & mFITEXY OLS$(Y|X)$ & $9.06^{+0.09}_{-0.08}$ & $-0.26^{+0.07}_{-0.08}$ & $-25.73$ & $0.37$ & $0.42$ \\
                   & mFITEXY OLS$(X|Y)$ & $9.0^{+0.1}_{-0.1}$ & $-0.7^{+0.2}_{-0.3}$ & $-25.73$ & $0.85$ & $0.67$ \\
                   & mFITEXY Bisector   & $9.0^{+0.1}_{-0.1}$ & $-0.5^{+0.1}_{-0.2}$ & $-25.73$ & $-$    & $0.48$ \\
                   & Kelly OLS$(Y|X)$  & $9.0 \pm 0.1$ & $-0.24 \pm 0.09$ & $-25.73$ & $0.41 \pm 0.09$ & $0.42$ \\
                   & Kelly OLS$(X|Y)$  & $$ & $$ & $-25.73$ & $$ & $$ \\
                   & Kelly Bisector    & $$ & $$ & $-25.73$ & $-$    & $$ \\

S\'ersic (44) & BCES OLS$(Y|X)$   & $7.71 \pm 0.09$ & $-0.41 \pm 0.08$ & $-22.92$ & $-$    & $0.61$ \\
              & BCES OLS$(X|Y)$   & $7.7  \pm 0.1$  & $-0.9  \pm 0.2 $ & $-22.92$ & $-$    & $0.93$ \\
              & BCES Bisector     & $7.7  \pm 0.1$  & $-0.61 \pm 0.08$ & $-22.92$ & $-$    & $0.71$ \\
              & mFITEXY OLS$(Y|X)$ & $7.72^{+0.08}_{-0.08}$ & ${-0.41}^{+0.07}_{-0.07}$ & $-22.92$ & $0.54$ & $0.61$ \\
              & mFITEXY OLS$(X|Y)$ & $7.7^{+0.1}_{-0.1}$ & $-0.9^{+0.1}_{-0.2}$ & $-22.92$ & $0.89$ & $0.93$ \\
              & mFITEXY Bisector   & $7.7^{+0.1}_{-0.1}$ & $-0.61^{+0.09}_{-0.12}$ & $-22.92$ & $-$     & $0.71$ \\
              & Kelly OLS$(Y|X)$  & $7.72 \pm 0.09$ & $-0.41 \pm 0.09$ & $-22.92$ & $0.55 \pm 0.08$ & $0.61$ \\
              & Kelly OLS$(X|Y)$  & $7.7 \pm  0.1$ & $-0.9 \pm  0.2$ & $-22.92$ & $0.8 \pm  0.3$ & $0.98$ \\
              & Kelly Bisector    & $$ & $$ & $-22.92$ & $-$    & $$ \\

%Ellipticals (E) (30)   & BCES OLS$(Y|X)$   & $8.80 \pm 0.07$ & $-0.53 \pm 0.09$ & $-25.45$ & $-$    & $0.42$ \\
%                      & BCES OLS$(X|Y)$   & $8.80 \pm 0.08$ & $-0.66 \pm 0.07$ & $-25.45$ & $-$    & $0.48$ \\
%                      & BCES Bisector     & $8.80 \pm 0.08$ & $-0.59 \pm 0.07$ & $-25.45$ & $-$    & $0.44$ \\
%                      & FITEXY OLS$(Y|X)$ & $8.81^{+0.07}_{-0.07}$ & $-0.44^{+0.06}_{-0.07}$ & $-25.45$ & $0.33$     & $0.40$ \\
%                      & FITEXY OLS$(X|Y)$ & $8.78^{+0.09}_{-0.09}$ & $-0.7^{+0.1}_{-0.1}$    & $-25.45$ & $0.60$     & $0.50$\\
%                      & FITEXY Bisector   & $8.80^{+0.08}_{-0.08}$ & $-0.56^{+0.08}_{-0.10}$ & $-25.45$ & $-$        & $0.43$ \\
%
%Lenticulars (S0) (13)  & BCES OLS$(Y|X)$   & $7.9 \pm 0.1$ & $-0.4 \pm 0.2$ & $-22.19$ & $-$    & $0.56$ \\
%                      & BCES OLS$(X|Y)$   & $7.9 \pm 0.3$ & $-1.1 \pm 0.5$ & $-22.19$ & $-$    & $1.01$ \\
%                      & BCES Bisector     & $7.9 \pm 0.2$ & $-0.7 \pm 0.2$ & $-22.19$ & $-$    & $0.69$ \\
%                      & FITEXY OLS$(Y|X)$ & $7.9^{+0.2}_{-0.1}$ & $-0.3^{+0.2}_{-0.2}$ & $-22.19$ & $0.51$     & $0.56$ \\
%                      & FITEXY OLS$(X|Y)$ & $7.8^{+0.3}_{-0.4}$ & $-1.3^{+0.4}_{-1.3}$ & $-22.19$ & $0.71$     & $1.20$\\
%                      & FITEXY Bisector   & $7.9^{+0.2}_{-0.3}$ & $-0.7^{+0.3}_{-0.4}$ & $-22.19$ & $-$        & $0.71$ \\
%
{\bf Early-type (E+S0)} (45) & BCES OLS$(Y|X)$    & $8.56 \pm 0.07$ & $-0.33 \pm 0.04$ & $-24.47$ & $-$    & $0.46$ \\
                             & BCES OLS$(X|Y)$    & $8.56 \pm 0.08$ & $-0.48 \pm 0.05$ & $-24.47$ & $-$    & $0.55$ \\
                             & {\bf BCES Bisector}& $\boldsymbol{8.56 \pm 0.07}$ & $\boldsymbol{-0.40 \pm 0.04}$ & $\boldsymbol{-24.47}$ & $-$    & $\boldsymbol{0.49}$ \\
                             & mFITEXY OLS$(Y|X)$ & $8.56^{+0.06}_{-0.06}$ & $-0.32^{+0.03}_{-0.04}$ & $-24.47$ & $0.40$ & $0.46$ \\
                             & mFITEXY OLS$(X|Y)$ & $8.54^{+0.08}_{-0.08}$ & $-0.49^{+0.05}_{-0.06}$ & $-24.47$ & $1.00$ & $0.57$\\
                             & mFITEXY Bisector   & $8.55^{+0.07}_{-0.07}$ & $-0.41^{+0.04}_{-0.05}$ & $-24.47$ & $-$    & $0.49$ \\
                             & Kelly OLS$(Y|X)$  & $8.55 \pm 0.07$ & $-0.32 \pm 0.04$ & $-24.47$ & $0.42 \pm 0.06$ & $ 0.46$ \\
                             & Kelly OLS$(X|Y)$  & $8.55 \pm 0.09$ & $-0.49 \pm 0.07$ & $-24.47$ & $0.5 \pm  0.1$ & $0.57$ \\
                             & Kelly Bisector    & $$ & $$ & $-24.47$ & $-$    & $$ \\

{\bf Late-type (Sp)} (17) & BCES OLS$(Y|X)$    & $7.2 \pm 0.2$ & $-0.8 \pm 0.4$ & $-22.33$ & $-$    & $0.70$ \\
                          & BCES OLS$(X|Y)$    & $7.2 \pm 0.3$ & $-1.7 \pm 0.7$ & $-22.33$ & $-$    & $1.26$ \\
                          & {\bf BCES Bisector}& $\boldsymbol{7.2 \pm 0.2}$ & $\boldsymbol{-1.1 \pm 0.3}$ & $\boldsymbol{-22.33}$ & $-$    & $\boldsymbol{0.88}$ \\
                          & mFITEXY OLS$(Y|X)$ & $7.2^{+0.1}_{-0.1}$ & $-0.5^{+0.2}_{-0.2}$ & $-22.33$ & $0.54$ & $0.63$ \\
                          & mFITEXY OLS$(X|Y)$ & $7.4^{+0.5}_{-0.3}$ & $-2.0^{+0.7}_{-2.3}$ & $-22.33$ & $0.54$ & $1.51$ \\
                          & mFITEXY Bisector   & $7.3^{+0.4}_{-0.3}$ & $-1.0^{+0.3}_{-0.5}$ & $-22.33$ & $-$    & $0.82$ \\
                          & Kelly OLS$(Y|X)$  & $7.2 \pm 0.2$ & $-0.5 \pm 0.3$ & $-22.33$ & $0.6 \pm 0.2$ & $0.62$ \\
                          & Kelly OLS$(X|Y)$  & $$ & $$ & $-22.33$ & $$ & $$ \\
                          & Kelly Bisector    & $$ & $$ & $-22.33$ & $-$    & $$ \\

\hline 
\\
 & \multicolumn{6}{l}{\emph{Black hole mass -- galaxy luminosity}} \\
  & \multicolumn{6}{l}{$\log[M_{\rm BH}/{\rm M_\odot}] = \alpha + \beta[(MAG_{\rm gal} - X_0)/{\rm mag}]$} \\ [0.5em]
All (62)               & BCES OLS$(Y|X)$   & $8.26 \pm 0.08$ & $-0.49 \pm 0.06$ & $-24.78$ & $-$ & $0.64$ \\
                       & BCES OLS$(X|Y)$   & $8.3 \pm 0.1$   & $-1.0 \pm 0.1$   & $-24.78$ & $-$ & $0.92$ \\
                       & BCES Bisector     & $8.26 \pm 0.09$ & $-0.72 \pm 0.07$ & $-24.78$ & $-$ & $0.71$ \\
                       & mFITEXY OLS$(Y|X)$ & $8.26^{+0.08}_{-0.08}$ & $-0.48^{+0.06}_{-0.07}$ & $-24.78$ & $0.62$ & $0.64$ \\
                       & mFITEXY OLS$(X|Y)$ & $8.3^{+0.1}_{-0.1}$    & $-1.0^{+0.1}_{-0.2}$    & $-24.78$ & $0.87$ & $0.93$ \\
                       & mFITEXY Bisector   & $8.3^{+0.1}_{-0.1}$    & $-0.72^{+0.08}_{-0.10}$ & $-24.78$ & $-$    & $0.71$ \\
                       & Kelly OLS$(Y|X)$  & $8.26 \pm 0.08$ & $-0.48 \pm 0.07$ & $-24.78$ & $0.63 \pm 0.06$ & $0.64$ \\
                       & Kelly OLS$(X|Y)$  & $8.3 \pm 0.1$ & $-1.0 \pm 0.2$ & $-24.78$ & $0.9 \pm 0.2$ & $0.96$ \\
                       & Kelly Bisector    & $$ & $$ & $-24.78$ & $-$    & $$ \\
                   
Early-type (E+S0) (45) & BCES OLS$(Y|X)$   & $8.56 \pm 0.06$ & $-0.43 \pm 0.05$ & $-24.88$ & $-$ & $0.45$ \\
                       & BCES OLS$(X|Y)$   & $8.56 \pm 0.08$ & $-0.63 \pm 0.05$ & $-24.88$ & $-$ & $0.53$ \\
                       & BCES Bisector     & $8.56 \pm 0.07$ & $-0.53 \pm 0.04$ & $-24.88$ & $-$ & $0.47$ \\
                       & mFITEXY OLS$(Y|X)$ & $8.56^{+0.06}_{-0.06}$ & $-0.41^{+0.05}_{-0.05}$ & $-24.88$ & $0.41$ & $0.45$ \\
                       & mFITEXY OLS$(X|Y)$ & $8.56^{+0.07}_{-0.09}$ & $-0.66^{+0.06}_{-0.08}$ & $-24.88$ & $0.79$ & $0.55$ \\
                       & mFITEXY Bisector   & $8.56^{+0.07}_{-0.07}$ & $-0.53^{+0.05}_{-0.07}$ & $-24.88$ & $-$    & $0.47$ \\
                       & Kelly OLS$(Y|X)$  & $8.56 \pm 0.07$ & $-0.42 \pm 0.06$ & $-24.88$ & $0.44 \pm 0.06$ & $0.45$ \\
                       & Kelly OLS$(X|Y)$  & $8.56 \pm 0.09$ & $-0.66 \pm 0.09$ & $-24.88$ & $0.6 \pm 0.1$ & $0.55$ \\
                       & Kelly Bisector    & $$ & $$ & $-24.88$ & $-$    & $$ \\
                   
%All (62)               & BCES OLS$(Y|X)$   & $$ & $$ & $$ & $-$ & $$ \\
%                       & BCES OLS$(X|Y)$   & $$ & $$ & $$ & $-$ & $$ \\
%                       & BCES Bisector     & $$ & $$ & $$ & $-$ & $$ \\
%                       & FITEXY OLS$(Y|X)$ & $$ & $$ & $$ & $$ & $$ \\
%                       & FITEXY OLS$(X|Y)$ & $$ & $$ & $$ & $$ & $$ \\
%                       & FITEXY Bisector   & $$ & $$ & $$ & $-$    & $$ \\
%                       & Kelly OLS$(Y|X)$  & $$ & $$ & $$ & $$ & $$ \\
%                       & Kelly OLS$(X|Y)$  & $$ & $$ & $$ & $$ & $$ \\
%                       & Kelly Bisector    & $$ & $$ & $$ & $-$    & $$ \\

\hline 
\\
 & \multicolumn{6}{l}{\emph{Black hole mass -- spheroid stellar mass}} \\
  & \multicolumn{6}{l}{$\log[M_{\rm BH}/{\rm M_\odot}] = \alpha + \beta \log[(M_{\rm *,sph} - X_0)/{\rm M_\odot}]$} \\ [0.5em]
{\bf Early-type (E+S0)} (45)  & BCES OLS$(Y|X)$    & $8.56 \pm 0.07$ & $0.8 \pm 0.1$ & $10.81$ & $-$ & $0.48$ \\
                              & BCES OLS$(X|Y)$    & $8.56 \pm 0.08$ & $1.3 \pm 0.1$ & $10.81$ & $-$ & $0.59$ \\
                              & {\bf BCES Bisector}& $\boldsymbol{8.56 \pm 0.07}$ & $\boldsymbol{1.0 \pm 0.1}$ & $\boldsymbol{10.81}$ & $-$ & $\boldsymbol{0.51}$ \\
                              & FITEXY OLS$(Y|X)$  & $$ & $$ & $10.81$ & $$ & $$ \\
                              & FITEXY OLS$(X|Y)$  & $$ & $$ & $10.81$ & $$ & $$ \\
                              & FITEXY Bisector    & $$ & $$ & $10.81$ & $-$    & $$ \\
                              & Kelly OLS$(Y|X)$  & $8.55 \pm 0.07$ & $0.8 \pm 0.1$ & $10.81$ & $0.43 \pm 0.06$ & $0.48$ \\
                              & Kelly OLS$(X|Y)$  & $8.55 \pm 0.09$ & $1.3 \pm  0.2$ & $10.81$ & $0.6 \pm  0.1$ & $0.61$ \\
                              & Kelly Bisector    & $$ & $$ & $10.81$ & $-$    & $$ \\

{\bf Late-type (Sp)} (17)    & BCES OLS$(Y|X)$    & $7.2 \pm 0.2$ & $1.9 \pm 1.5$ & $10.05$ & $-$ & $0.74$ \\
                             & BCES OLS$(X|Y)$    & $7.2 \pm 0.4$ & $5.9 \pm 3.4$ & $10.05$ & $-$ & $1.70$ \\
                             & {\bf BCES Bisector}& $\boldsymbol{7.2 \pm 0.2}$ & $\boldsymbol{3.0 \pm 1.3}$ & $\boldsymbol{10.05}$ & $-$ & $\boldsymbol{0.94}$ \\
                             & FITEXY OLS$(Y|X)$  & $$ & $$ & $10.05$ & $$ & $$ \\
                             & FITEXY OLS$(X|Y)$  & $$ & $$ & $10.05$ & $$ & $$ \\
                             & FITEXY Bisector    & $$ & $$ & $10.05$ & $-$    & $$ \\
                             & Kelly OLS$(Y|X)$  & $7.2 \pm 0.2$ & $0.9 \pm 1.0$ & $10.05$ & $0.7 \pm 0.2$ & $0.65$ \\
                             & Kelly OLS$(X|Y)$  & $$ & $$ & $10.05$ & $$ & $$ \\
                             & Kelly Bisector    & $$ & $$ & $10.05$ & $-$    & $$ \\
                   
                  
\hline 
\hline
\end{tabular}
\label{tab:lreg} 
\end{table*}


\section{Analysis}
\label{sec:anal}
We performed a linear regression analysis of the $M_{\rm BH} - L_{\rm sph}$, $M_{\rm BH} - M_{\rm *,sph}$ and $M_{\rm BH} - L_{\rm gal}$ diagrams 
using the BCES code from \cite{akritasbershady1996} 
and also the FITEXY routine \citep{press1992} 
modified by \cite{tremaine2002} to account for the intrinsic vertical scatter, $\epsilon$. 
Because the BCES routine {\bf AL: did you wanna write (correctly) this sentence for me please? 
can give biased results (wrong slope) when the scatter is comparable to the range of the data},
we checked that the best-fit parameters obtained with the BCES code were consistent with those output by the modified FITEXY code.
In Table \ref{tab:lreg} we report the BCES and modified FITEXY linear regressions, both symmetrical and non-symmetrical, 
for S\'ersic/core-S\'ersic galaxies and for different galaxy morphological types (elliptical/lenticular, spiral).
Symmetrical regressions are meant to be compared with theoretical expectations, 
whereas non-symmetrical Ordinary Least Squares $(Y|X)$ regressions -- 
since they minimize the scatter in the vertical direction -- 
can be used to predict black hole masses.

\section{Results and discussion}
\label{sec:res}

\subsection{Black hole mass -- spheroid luminosity}
We show the $M_{\rm BH} - L_{\rm sph}$ diagram in Figure \ref{fig:mbhmagsph}. \\
S\'ersic and core-S\'ersic spheroids have slopes consistent with each other (within their $1\sigma$ uncertainties), 
in disagreement with the findings of GS13. 
The slope that we obtained for core-S\'ersic spheroids ($M_{\rm BH} \propto L_{\rm sph}^{1.2 \pm 0.2}$) 
is consistent with the slope reported by GS13 in the $K_s$-band ($M_{\rm BH} \propto L_{\rm sph}^{1.1 \pm 0.2}$). 
Instead, the slope that we determined for S\'ersic spheroids ($M_{\rm BH} \propto L_{\rm sph}^{1.5 \pm 0.2}$) 
is significantly shallower than that found by GS13 ($M_{\rm BH} \propto L_{\rm sph}^{2.7 \pm 0.5}$). 
Although the S\'ersic/core-S\'ersic classification used by GS13 slightly differs\footnote{which galaxies?} from the classification used here, 
the main cause of such inconsistency is that the bulge-to-total ratios obtained from our galaxy decompositions 
are different from those assumed by GS13 to convert galaxy luminosities into bulge luminosities.
Our bulge-to-total ratios for low-luminosity S\'ersic spheroids ($MAG_{\rm sph} \gtrsim -22 \rm~mag$) 
are smaller than those used by GS13. 
The host galaxies of such bulges are late-type, spiral galaxies, 
which typically present a complex morphology (bars, double bars, embedded disks, nuclear components, etc).
Our sophisticated galaxy models account for the extra components, 
while the bulge-to-total ratios of GS13 were derived from simple bulge/disk decompositions 
which overestimated the bulge luminosity.
This results in our bulge magnitudes being on average $\sim$$1\rm~mag$ fainter.
On the other side, our bulge-to-total ratios for high-luminosity S\'ersic spheroids ($MAG_{\rm sph} \lesssim -24 \rm~mag$) 
are on average larger than those adopted by GS13.
In this case, the host systems are early-type elliptical/lenticular galaxies that feature intermediate-scale disks\footnote{explain}.
Past bulge/disk decompositions failed to correctly identify the extent of such disks and treated them as large-scale disks, 
thus underestimating the bulge luminosity.
The magnitudes that we obtained for such spheroids are on average $\sim$$1\rm~mag$ brighter. \\
We have seen that the change in slope of the $M_{\rm BH} - L_{\rm sph}$ correlation -- 
which is expected for consistency with other scaling relations 
(a single power-law $M_{\rm BH} - \sigma$ correlation and a double power-law $L_{\rm sph} - \sigma$ correlation) -- 
cannot be attributed to the division between the two populations of S\'ersic and core-S\'ersic spheroids.
We now test a new hypothesis, that is to say the change in slope is to be ascribed to the different formation mechanisms of early- and late-type galaxies. 
If this hypothesis is correct, 
the spheroids of early-type (elliptical and lenticular) galaxies will follow $M_{\rm BH} \propto L_{\rm sph}^{\sim 1}$, 
whereas the spheroids of late-type (spiral) galaxies will have $M_{\rm BH} \propto L_{\rm sph}^{\sim 2.5}$.
First, we checked that the slopes of the bulges of elliptical and lenticular galaxies, separately, are consistent with each other, 
and thus, together, they define a single \emph{red sequence} in the $M_{\rm BH} - L_{\rm sph}$ diagram. 
We then fit the bulges of early- and late-type galaxies with two separate log-linear regressions, 
and obtained $M_{\rm BH} \propto L_{\rm sph}^{1 \pm 0.1}$ and $M_{\rm BH} \propto L_{\rm sph}^{2.75 \pm 0.75}$ respectively, 
in excellent agreement with the theoretical expectations of our hypothesis.
{\bf vertical scatter along the early type sequence is constant?
if 2 overmassive bhs removed, what happens?}

{\bf coming now to figure 1: if I look at the blue points it is difficult to believe that they define a linear relation, 
their magnitude range is too small and it could just be that at low galaxy luminosities the scatter of the MBH-L relation increases. 
This is what most models predict. Do the galaxies below the correlation have some special property? 
Could it just be that spiral bulges simple do not follow a tight relation with BH mass? 

Giulia, I think you already plan to do this, but can you include the error bars in all places where a relation is presented, so that readers know if there is a relation or not.
Remember also Ewan's remark about the unsuitability of the Pearson and Spearman correlation coefficients because they do not take into account the error bars on our data - 
so don't include those simple coefficients.  Actually, it would be worth pointing out to readers that a visual inspection of the plotted data requires 
the viewer to take into account the error bars when judging-by-eye the strength of a correlation. 
Maybe then mention in the following sentence that this is why the correlation coefficients are not applicable here. }

In Figure \ref{fig:pseudob}, we show the distribution of spheroid S\'ersic indices\footnote{The spheroid S\'ersic indices 
are taken from our one-dimensional fits of the galaxy major-axis surface brightness profiles (\emph{Paper I}).}, 
$n_{\rm sph}$, in the $M_{\rm BH} - L_{\rm sph}$ diagram.
Our aim is to check whether bulges with $n_{\rm sph}<2$, claimed by some to be pseudobulges (e.g.~ cite), 
are offset to lower black hole masses from the correlation defined by (classical) bulges with $n_{\rm sph}>2$, 
as reported by \cite{sani2011} {\bf who else?}. 
To do this, we fit a symmetrical linear regression to the bulges that have $n_{\rm sph}>2$ 
and we compute the vertical offset of all bulges from the regression. 
In the inset of Figure \ref{fig:pseudob}, we plot the vertical offset against $n_{\rm sph}$. 
From a visual inspection, bulges with $n_{\rm sph}<2$ do not appear to be offset from the correlation traced by bulges with $n_{\rm sph}>2$.

{\bf As you probably know, and as explained in many of my papers (e.g. http://arxiv.org/abs/1501.02937 ; http://arxiv.org/abs/1108.0997 ; http://arxiv.org/abs/1311.7207 ), 
pseudobulges are notoriously hard to identify.  It would be a mistake to have a column labelled "pseudobulge".   
However a galaxy classification column, indicating the presence of a bar, is good to include.  
While we do not wish to publish the Sersic indices here (that will be kept for a different paper), 
we could have a column indicating if n is above or below 2 - given that some people mistakingly think that they can use this as a dividing point. 
Giulia, can you also include a nearby reference to Graham (2013, http://arxiv.org/abs/1108.0997) so that people can read and understand why $n < 2$ 
does not necessarily equate to a pseudobulge. }


\subsection{Black hole mass -- galaxy luminosity}
Figure \ref{fig:mbhmaggal} illustrates the $M_{\rm BH} - L_{\rm gal}$ diagram.
Four spiral galaxies had their total luminosities underestimated (see Section \ref{sec:data}) 
and thus are not included in the linear regression analysis. 
\cite{lasker2014anal} 

Contrary to a previous claim, the bulge luminosity appears to be a better indicator of the BH mass than the galaxy luminosity.

compare scatter mbh-msph and mbh-mgal for early types
intrinsic scatter of early type gals in m-mbul is slightly more than in m-mgal, but this might just depend 
on the ability of measuring the bulge luminosity, which is model dependent, the gal lum is not.
only for early type galaxies one can use the tot luminosity as bh mass indicator

errors on gal mag??????????????? 0.2 constant 

\subsection{Black hole mass -- spheroid stellar mass}
Finally, we present the $M_{\rm BH} - M_{\rm *,sph}$ diagram in Figure \ref{fig:mbhmasssph}, 
where we include our sample of 66 galaxies, 
plus the 35 + 139 low-mass AGNs taken from the compilation of \cite{grahamscott2015}. 

addition of agns
relations
agns consistent w spirals, but analysis deferred to other paper
cubic growth


\section{Conclusions}
\label{sec:concl}



\acknowledgments
%GS warmly thanks Chieng Peng, Peter Erwin, Luca Cortese, Elisabete Lima Da Cunha and Gonzalo Diaz 
%for useful discussion. \\
This research was supported by Australian Research Council funding through grants
DP110103509 and FT110100263.
This work is based on observations made with the IRAC instrument \citep{fazio2004IRAC} 
on-board the Spitzer Space Telescope, 
which is operated by the Jet Propulsion Laboratory, 
California Institute of Technology under a contract with NASA.
This research has made use of the GOLDMine database \citep{goldmine} and the NASA/IPAC Extragalactic Database (NED) 
which is operated by the Jet Propulsion Laboratory, California Institute of Technology, 
under contract with the National Aeronautics and Space Administration. 


\bibliography{/Users/gsavorgnan/Dropbox_notsync/giulia_e_basta/Literature/SMBHbibliography}


\clearpage


\end{document}

