\chapter{$M_{\rm BH} - n_{\rm sph}$}
\label{ch:mn}

While the previous Chapter was dedicated to the study of the relations between 
black hole mass and galaxy or spheroid luminosity, 
in this Chapter we perform an analogous analysis 
to explore the relation between black hole mass and spheroid S\'ersic index ($M_{\rm BH} - n_{\rm sph}$). 
To address the issue of consistency between galaxy scaling relations 
that was outlined in Chapter \ref{ch:intro}, 
it is mandatory to include also an analysis of the relation between spheroid luminosity 
and spheroid S\'ersic index ($L_{\rm sph} - n_{\rm sph}$). 
The reliability of the uncertainties associated with the S\'ersic index measurements 
obtained from our 1D decompositions 
ensures a robust estimate of the intrinsic scatter in the $M_{\rm BH} - n_{\rm sph}$ diagram, 
which can be compared with that in the $M_{\rm BH} - L_{\rm sph}$ diagram. \\ 

From a physical point of view, the $M_{\rm BH} - n_{\rm sph}$ correlation is interesting because 
the S\'ersic index is a measure of the central radial concentration of stars, 
which dictates the radial distribution of mass within a galaxy's spheroid 
and therefore determines the dynamical response of stars 
as measured through the observable $\sigma_*$ \citep{grahamdriver2007}. \\

The remainder of this Chapter comprises the published version of the paper 
``Supermassive Black Holes and Their Host Spheroids. 
III. The $M_{\rm BH} - n_{\rm sph}$ correlation'' 
by G.~A.~D.~Savorgnan,  
as it appears in Volume 821 of the \emph{The Astrophysical Journal}. 



\includepdf[pages={1-8}]{ApJ2016b.pdf}
