\chapter*{Abstract}

Supermassive black holes in the local Universe obey a surprisingly large number of scaling laws 
that involve the black hole mass and various properties of the host spheroids. 
These ``black hole mass scaling relations'' reveal a strong symbiosis between galaxies and black holes, 
define important constraints about their co-evolution through the cosmic time, 
and set the boundary conditions (at $z=0$) for theoretical models and simulations of galaxy evolution. 

A careful galaxy decomposition is required to accurately measure a galaxy's spheroid properties. 
Recent studies have performed structural decomposition 
for similar samples of galaxies with a direct measure of the black hole mass, 
but they have not converged to the same conclusions. 
This is because their best-fit model parameters for the same galaxy were often significantly different 
due to various systematic effects, 
and their models for the same galaxy were frequently not consistent with each other in terms of fitted components. 
Moreover, none of these studies attempted an individual galaxy-by-galaxy comparison of their models 
with the previous literature. 
We have now made this comparison, identified the optimal decomposition, 
and obtained improved black hole mass scaling relations. 

Using \emph{Spitzer} observations at $3.6 ~\mu \rm m$, 
which is the best available wavelength band to trace the stellar mass, 
we performed state-of-the-art structural decompositions for $66$ galaxies 
with a direct measure of their black hole mass. 
Thanks to a meticulous inspection of each galaxy's substructure -- 
through photometric and isophotal analysis, unsharp masking, auxiliary information extracted from the literature, 
and, for the first time, kinematic maps, 
we are able to identify \emph{a priori} the physical galaxy components. 
Our multicomponent models account for spheroid, large-scale disc, 
embedded or nuclear discs, spiral arms, bars, rings, halo, extended or unresolved nuclear source and partially depleted core. 
The combination of photometric and kinematic information was crucial 
to confirm the presence of rotationally supported components in most early-type (elliptical+lenticular) galaxies, 
and to identify their radial extent (nuclear, intermediate-scale or large-scale discs). 
Upon performing galaxy decompositions with both one-dimensional (1D) and two-dimensional (2D) parametric techniques, 
we observed no systematic differences between the results from 1D and 2D methods, 
but we found more advantages in the former. 
















%Our black hole mass range spans 5 orders of magnitude and allows us to explore the less studied low-mass end of the correlations. 
%We use Spitzer/IRAC 3.6  mosaics because the 3.6  luminosity is the best tracer of the stellar mass.

We confirm that S\'ersic and core-S\'ersic galaxies follow different trends and therefore are associated to different evolutionary scenarios. 


We present updates and modifications to several key black hole mass scaling relations, 
and discuss the important implications for galaxy evolution models. 



Unlike previous studies that mainly investigated high-mass, early-type galaxies, 
our sample contains a large number of spiral galaxies with low black hole masses ($<10^7 M$), 
and allows us to better investigate the poorly studied low-mass end of the black hole scaling relations.
We present updates and modifications to several key black hole mass scaling relations, showing that early-type (elliptical+lenticular) 
and late-type (spiral) galaxies follow different trends and thus their black holes grow at a different pace, 
and we discuss the important implications for galaxy evolution models. 

