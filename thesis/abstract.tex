\chapter*{Abstract}

Supermassive black holes in the local Universe obey a surprisingly large number of scaling laws 
that involve the black hole mass and various properties of the host spheroids. 
These ``black hole mass scaling relations'' reveal a strong symbiosis between galaxies and black holes, 
define important constraints about their co-evolution through the cosmic time, 
and set the boundary conditions (at $z=0$) for theoretical models and simulations of galaxy evolution. \\

%%A careful galaxy decomposition is required to accurately measure a galaxy's spheroid properties. 
%%Recent studies have attempted structural decomposition 
%%for similar samples of galaxies with a direct measure of the black hole mass, 
%%but they have not converged to the same conclusions. 
%%This is because their models for the same galaxy were often not consistent with each other in terms of fitted components, 
%%or, even when they used the same model for the same galaxy, they obtained significantly different best-fit parameters 
%%due to various systematic effects.
%%Moreover, none of these studies attempted an individual galaxy-by-galaxy comparison of their models 
%%with the previous literature. 
%%We have now made this comparison, identified the optimal decomposition, 
%%and obtained improved black hole mass scaling relations, 
%%which are detailed in this thesis. \\

Using \emph{Spitzer} observations at $3.6 ~\mu \rm m$, 
which is the best available wavelength band to trace the stellar mass, 
we performed state-of-the-art structural decompositions for $66$ galaxies 
with a direct measure of their black hole mass. 
Thanks to a meticulous inspection of each galaxy's substructure -- 
through photometric and isophotal analysis, unsharp masking, auxiliary information extracted from the literature, 
and, for the first time, kinematic maps -- 
we were able to identify \emph{a priori} the physical galaxy components. 
%Our multicomponent models account for spheroid, large-scale disc, 
%embedded or nuclear discs, spiral arms, bars, rings, halo, extended or unresolved nuclear source and partially depleted core. 
The combination of photometric and kinematic information was crucial 
to confirm the presence of rotationally supported components in most early-type (elliptical+lenticular) galaxies, 
and to identify their radial extent (nuclear, intermediate-scale or large-scale discs). 
Upon performing galaxy decompositions with both one-dimensional (1D) and two-dimensional (2D) parametric techniques, 
we observed no systematic differences between the results from 1D and 2D methods, 
but we found more advantages in the former. 
Our 66 galaxies constitute the largest sample to date 
for which these two conditions are satisfied: 
(\emph{i}) the spheroid structural parameters have been measured accurately and homogeneously;  
(\emph{ii}) and the black hole mass has been securely estimated with a direct method. 
%Unlike previous studies that mainly investigated massive early-type galaxies, 
%our sample contains a large number of spiral galaxies with low black hole masses ($<10^7 \rm~M_\odot$), 
%and allowed us to better investigate the poorly studied low-mass end of some black hole mass correlations. 
This thesis presents updates and modifications to several black hole mass scaling relations, 
and discusses important implications for galaxy evolution models. \\

mm + mn \\

ubermassive \\


