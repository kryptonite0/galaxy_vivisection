\chapter*{Abstract}

Supermassive black holes in local galaxies obey a surprisingly large number of scaling laws 
that involve the black hole mass and various properties of the host spheroid or galaxy. 
These ``black hole mass scaling relations'' reveal a strong symbiosis between galaxies and black holes, 
define important constraints about their co-evolution through the cosmic time, 
and set the boundary conditions (at $z=0$) for theoretical models and simulations of galaxy evolution. 

Using \emph{Spitzer} observations at $3.6~\mu\rm m$, 
which is the best available wavelength band to trace the stellar mass, 
we performed state-of-the-art structural decompositions for $66$ galaxies 
with a direct measure of their black hole mass, $M_{\rm BH}$. 
Thanks to a meticulous inspection of each galaxy's substructure -- 
by means of photometric and isophotal analysis, unsharp masking, auxiliary information extracted from the literature, 
and, for the first time, kinematic maps -- 
we were able to identify \emph{a priori} the physical galaxy components. 
The combination of photometric and kinematic information was crucial 
to confirm the presence of rotationally supported components in most early-type (elliptical+lenticular) galaxies, 
and to identify their radial extent (nuclear, intermediate-scale or large-scale discs). 
Upon performing galaxy decompositions with both one-dimensional (1D) and two-dimensional (2D) parametric techniques, 
we observed no systematic differences between the results from 1D and 2D methods, 
but we found more advantages in the former. 
Our 66 galaxies constitute the largest sample to date 
for which the following two conditions are satisfied: 
(\emph{i}) the spheroid structural parameters have been measured accurately and homogeneously;  
(\emph{ii}) and the black hole mass has been securely estimated with a direct method. 
This thesis presents updates and modifications to several black hole mass scaling relations, 
and discusses important implications for galaxy evolution models. 

For early-type (elliptical + lenticular) galaxies only, 
the black hole mass correlates equally well with galaxy luminosity, $L_{\rm gal}$, 
and spheroid luminosity, $L_{\rm sph}$, or spheroid stellar mass, $M_{\rm *,sph}$. 
However, when all galaxies (early- and late-type) are considered together, 
the $M_{\rm BH} - L_{\rm sph}$ relation has a lower level of intrinsic scatter than the $M_{\rm BH} - L_{\rm gal}$ relation. 
In the $M_{\rm BH} - M_{\rm *,sph}$ diagram, early-type galaxies define a tight linear correlation (a \emph{red sequence}), 
whereas late-type (spiral) galaxies follow a $2-3$ times steeper \emph{blue sequence}. 
The spheroid S\'ersic index, $n_{\rm sph}$, scales with $M_{\rm BH}$ 
in the same way for early- and late-type galaxies. 
Black holes that appear to be ``overmassive'' compared to expectations from the stellar velocity dispersion, $\sigma_*$, 
have been theoretically explained with their host galaxies having experienced more dry mergers than any other galaxy. 
In spite of that, we present empirical evidence supporting a scenario 
where the host galaxies of such overmassive black holes have undergone the lowest degree of dry merging. 
Finally, we debunk claims of four overmassive black holes in the $M_{\rm BH} - M_{\rm *,sph}$ diagram 
by demonstrating that the luminosity of their host spheroids had been considerably underestimated. 
We show that these four spheroids are unevolved relics of $z=2$ quiescent compact massive spheroids, 
and we confer about the significant consequences of this. 


