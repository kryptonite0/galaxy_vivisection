\chapter{Final remarks}
\label{ch:concl}

In this thesis, we explored scaling relations between the supermassive black hole mass 
and various properties of the host spheroid, 
aiming at gaining a more profound understanding of the co-evolution between SMBHs and galaxies. 
A summary of our principal findings and some additional brief considerations about them are presented here. 
The end of this Chapter incorporates some promising future research directions. \\

In Chapter \ref{ch:recov-mn} \citep{savorgnan2013}, we compared the results obtained by four independent studies 
\citep{grahamdriver2007,sani2011,vika2012,beifiori2012}
that attempted photometric decompositions of similar samples of galaxies with a direct measurement 
of the black hole mass. 
In many cases we found a large discrepancy between the S\'ersic index measurements 
obtained by different studies for the spheroidal component of the same galaxy, 
either due to a significantly different choice of model components 
or to other various systematic effects. 
By rejecting the most discrepant S\'ersic index measurements and averaging the remaining ones, 
we were able to recover a strong $M_{\rm BH} - n_{\rm sph}$ correlation, 
which was not found using the individual datasets of three of the four studies. 
From this we concluded that some of the galaxy decompositions were not accurate. 
Chapter \ref{ch:recov-mn} emphasises the importance of a correct, physically motivated selection of model components 
and cautions against the several systematics that can affect galaxy decomposition. \\

Chapter \ref{ch:galviv} was dedicated to the careful multicomponent decomposition of 66 galaxies 
with a direct measurement of the black hole mass. 
We followed the same approach as \citet{laurikainen2005} in selecting the components for each galaxy model.  
\emph{A priori} identification of the galaxy components was done on the basis of several different indicators 
such as the analysis of the isophotal parameters, 
the inspection of unsharp masks, 
complementary information extracted from the literature, 
and -- most importantly -- from the galaxy kinematics. 
A joint photometric-kinematic approach (e.g.~\citealt{krajnovic2013,arnold2014}) 
turned out to play a decisive role for the robustness of our galaxy modelling. 
Upon examining central \citep{atlas3dIII,scott2014} and more extended \citep{arnold2014} velocity maps 
and comparing kinematic and photometric signatures of stellar discs, 
we were able to securely identify the presence and the radial extent of such discs. 
We observed a wide range of spheroid-to-disc ratios in early-type galaxies, 
going from parsec-sized nuclear discs, to kiloparsec-sized intermediate- and large-scale discs, 
in agreement with the findings of \citet{krajnovic2013}. 
Comparison of our results with those obtained by previous studies indicates that 
multicomponent models (as opposed to simple bulge/disc models) are necessary 
to derive reliable structural parameters, confirming previous results 
(e.g.~\citealt{laurikainen2005,laurikainen2007,laurikainen2010,lasker2014data,salo2015}). 
In general, the best-fit parameters obtained with 1D and 2D decomposition techniques for the same galaxy 
are consistent with each other, 
i.e.~no systematic effects were noticed between 1D and 2D modelling. 
However, our practical experience led us to prefer the 1D decomposition technique. 
Advantages associated with the 1D technique are a higher convergence rate for the fits, 
the wealth of information contained in the 1D isophotal analysis, 
and the easier interpretation of the 1D residuals. 
We caution against the dangerous practice of identifying unsubtracted galaxy components 
from the residual image of a 2D fit. 
As an additional warning, 
given the level of detail to which our galaxy decompositions were carried out, 
we do not consider possible for current automatic routines to reproduce our analysis. \\

In Chapter \ref{ch:mm}, we used our dataset to derive and explore 
the $M_{\rm BH} - L_{\rm gal}$ and $M_{\rm BH} - L_{\rm sph}$ (or $M_{\rm BH} - M_{\rm *,sph}$) diagrams. 
When considering all galaxies, irrespective of their morphological type, 
the $M_{\rm BH} - L_{\rm sph}$ correlation has a lower level of intrinsic scatter 
than the $M_{\rm BH} - L_{\rm gal}$ correlation. 
However, when considering only early-type galaxies (elliptical+lenticular), 
$L_{\rm gal}$ and $L_{\rm sph}$ correlate equally well with $M_{\rm BH}$. 
\citet{lasker2014anal} found the same level of intrinsic scatter in their 
$M_{\rm BH} - L_{\rm gal}$ and $M_{\rm BH} - L_{\rm sph}$ diagrams for all galaxies
because their sample was highly biased towards early-type objects 
(among their 35 galaxies, only four are spirals). 
Our subsample of S\'ersic spheroids defines a less steep $M_{\rm BH} - L_{\rm sph}$ 
sequence than previously reported by \citet{grahamscott2013}. 
We do not observe any offset of spheroids with S\'ersic index $n_{\rm sph} < 2$ 
from the $M_{\rm BH} - L_{\rm sph}$ correlation traced by spheroids with $n_{\rm sph} > 2$, 
in disagreement with the claims of \citet{sani2011}. 
We identified two distinct trends in the $M_{\rm BH} - M_{\rm *,sph}$ diagram, 
that is, a \emph{red sequence} of early-type galaxies (elliptical+lenticular) 
following $M_{\rm BH} \propto M_{\rm *,sph}^{1.04 \pm 0.10}$ 
and a dramatically steeper \emph{blue sequence} of late-type galaxies (spiral) 
following $M_{\rm BH} \propto M_{\rm *,sph}^{2-3}$. 
This is in agreement with the near-linear $M_{\rm BH} - L_{\rm sph}$ (or $M_{\rm BH} - M_{\rm *,sph}$) correlation 
measured by previous studies that used galaxy samples dominated by high-mass early type objects 
(e.g.~\citealt{magorrian1998,marconihunt2003,haringrix2004,gultelkin2009,sani2011,beifiori2012,
erwingadotti2012,vika2012}),  
and it gives reason as to why other studies, 
whose galaxy samples gave a better representation of the late-type population 
with low-mass ($M_{\rm BH} \lesssim 10^8 \rm~M_\odot$) black holes, 
obtained a correlation steeper than linear 
or noticed super-linear deviations at the low-luminosity end of the diagram 
(e.g.~\citealt{laor1998,laor2001,wandel1999,salucci2000,ryan2007}). 
An interesting exception has been the under-linear $M_{\rm BH} \propto L_{\rm sph}^{0.75 \pm 0.10}$ 
measured by \citet{lasker2014anal}, 
who attributed the smaller log-slope of the correlation 
to the reduction of bulge luminosities derived from their multicomponents fits 
and to the type of linear regression algorithm used. 
It is worth mentioning that, among our 17 spiral galaxies, at least 12 host a Seyfert nucleus. 
\citet{wandel1999} and \citet{ryan2007} had previously noted that a linear $M_{\rm BH} - M_{\rm *,sph}$ relation 
systematically overestimates the $M_{\rm BH}/M_{\rm *,sph}$ ratio in Seyfert galaxies. 
In effect, the sample of $\approx 140$ low-redshift AGNs with virial black hole masses 
$10^5 \lesssim M_{\rm BH}/{\rm M_\odot} \lesssim 2 \times 10^6$ collected by \citet{grahamscott2015} 
appear to be the low-mass continuation of the $M_{\rm BH} - M_{\rm *,sph}$ sequence traced by our spiral galaxies. 
Finally, we note that our estimate of the normalization of the $M_{\rm BH} - M_{\rm *,sph}$ relation for early-type galaxies 
is the largest ever reported: $M_{\rm BH}/M_{\rm *,sph}$ is $\approx 0.7\%$ using our dataset, 
whereas it is as low as $\approx 0.5\%$ in \citet{scott2013} and \citet{kormendyho2013}, 
and even lower in previous studies. 
A recalibration of the $M_{\rm BH} - M_{\rm *,sph}$ normalization has important consequences 
for semi-analytical models and simulations of galaxy evolution that include black hole growth, 
and it also affects calculations of gravitational waves background from SMBH binary coalescence 
(but see Shankar et al.~\emph{submitted to MNRAS}). \\

m-n? \\

In Chapter \ref{ch:msigma}, we focused on the $M_{\rm BH} - \sigma$ diagram 
and in particular on the over-massive black holes that have been reported to exist 
at the high-mass end of the correlation. 
\citet{volontericiotti2013} explained the presence of the outlying black holes 
as a natural consequence of the fact that their host galaxies have experienced more dry mergers 
than any other galaxy, 
since dry mergers are expected to increase the black hole mass 
and ``conserve'' the stellar velocity dispersion. 
We tested and disproved this scenario for xx core-S\'ersic galaxies, 
using the ratio between black hole mass and stellar mass deficit as a proxy 
for the number of dry mergers experienced by the galaxy \citep{xxx}. 
This was confirmed by a second test using the central kinematic classification of xx galaxies. 
The slope of the m-sigma is this... 
comparison










 
