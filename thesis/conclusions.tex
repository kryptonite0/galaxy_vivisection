\chapter{Final Remarks and Future Perspectives}
\label{ch:concl}

In this thesis, we explored scaling relations between the supermassive black hole mass 
and various properties of the host spheroid, 
in the pursuit of a more profound understanding of the co-evolution between SMBHs and their host galaxies. 
A summary of our principal findings and some additional considerations are presented here. 
This Chapter incorporates also some promising future research directions. \\

In Chapter \ref{ch:recov-mn} \citep{savorgnan2013}, we compared the results obtained by four independent studies 
\citep{grahamdriver2007,sani2011,vika2012,beifiori2012}
that attempted photometric decompositions of similar samples of galaxies with a direct measurement 
of the black hole mass. 
In many cases we found a large discrepancy between the spheroid S\'ersic index measurements 
obtained by different studies for the same galaxy, 
either due to a significantly different choice of model components 
or to other various systematic effects. 
By rejecting the most discrepant S\'ersic index measurements and averaging the remaining ones, 
we were able to recover a strong $M_{\rm BH} - n_{\rm sph}$ correlation, 
which was not found using the individual datasets of three of the four aforementioned studies. 
From this we concluded that some of their galaxy decompositions were not accurate. 
Chapter \ref{ch:recov-mn} emphasises the importance of a correct, physically motivated selection of model components 
and cautions against the several systematics that can affect galaxy decomposition. \\

Chapter \ref{ch:galviv} \citep{paperI} was dedicated to the careful multicomponent decomposition of 66 galaxies 
with a direct measurement of the black hole mass. 
We followed the same approach as \citet{laurikainen2005} in selecting the components for each galaxy model.  
\emph{A priori} identification of the galaxy components was done on the basis of several different indicators 
such as the analysis of the isophotal parameters, 
the inspection of unsharp masks, 
complementary information extracted from the literature, 
and -- most importantly -- from the galaxy kinematics. 
A joint photometric-kinematic approach (e.g.~\citealt{krajnovic2013,arnold2014}) 
turned out to play a decisive role for the robustness of our galaxy modelling. 
Upon examining central \citep{atlas3dIII,scott2014} and more extended \citep{arnold2014} velocity maps 
and comparing kinematic and photometric signatures of stellar discs, 
we were able to securely identify the presence and the radial extent of such discs. 
We observed a wide range of spheroid-to-disc ratios in early-type galaxies, 
going from parsec-sized nuclear discs, to kiloparsec-sized intermediate- and large-scale discs, 
in agreement with the findings of \citet{krajnovic2013}. 
Comparison of our results with those obtained by previous studies indicates that 
multicomponent models (as opposed to simple bulge/disc models) are necessary 
to derive reliable structural parameters, confirming previous results 
(e.g.~\citealt{laurikainen2005,laurikainen2007,laurikainen2010,lasker2014data,salo2015}). 
In general, the best-fit parameters obtained with 1D and 2D decomposition techniques for the same galaxy 
are consistent with each other, 
i.e.~no systematic effects were noticed between 1D and 2D modelling. 
However, our practical experience led us to prefer the 1D decomposition technique. 
Advantages associated with the 1D technique are a higher convergence rate for the fits, 
the wealth of information contained in the 1D isophotal analysis, 
and the easier interpretation of the 1D residuals. 
We caution against the dangerous practice of identifying unsubtracted galaxy components 
from the residual image of a 2D fit. 
As an additional warning, 
given the level of detail to which our galaxy decompositions were carried out, 
we do not consider possible for current automatic routines to reproduce our human-supervised analysis. \\
In our 1D galaxy decomposition code, the minimisation routine is based on the Levenberg-Marquardt algorithm, 
which is known to be one of the fastest algorithms available, 
but is also prone to ``get trapped'' into local minima of the chi-squared distribution. 
As a future project, it would be interesting to implement in the code different minimisation algorithms, 
such as those available in {\tt Imfit} \citep{imfit},  
and test their performance. 
It would also be beneficial to create a procedure to estimate the systematic uncertainties 
associated with the best-fit parameters from the fit itself, 
without the need to resort to decompositions performed by other authors as we did in this work. 
One of the issues associated with multicomponent decomposition is the risk of ``overfitting'', 
i.e.~having a model with more components (or parameters) than what is actually needed. 
Information theory and Bayes analysis provide us with a number of methods 
to estimate the maximum number of parameters required by a model 
to fit a given dataset, 
such as the \citet{akaike1974} Information Criterion that was used in Chapter \ref{ch:mn}. 
One could check whether or not these methods agree with our human-supervised selection 
of ``physically motivated'' model components. 
Our structural analysis of galaxies confirmed the usefulness of a joint photometric-kinematic approach, 
especially when extended (i.e.~$> 1 R_{\rm e}$, \citealt{arnold2014}) kinematic maps are available. 
Future galaxy decomposition studies will be able to take advantage from the large datasets 
of on-going Integral Field Spectroscopy surveys (e.g.~SAMI, \citealt{croom2012}, and MaNGA, \citealt{law2014}). 
Finally, we note that \citet{ciambur2015} recently developed the IRAF task {\tt isofit}, 
which allows the user to perform a more sophisticated isophotal analysis of galaxies than the task {\tt ellipse} 
in the presence of an inclined disc. 
Our isophotal analysis was carried out before {\tt isofit} was conceived or available, 
but future studies will be able to use it. \\

In Chapter \ref{ch:mm} \citep{paperII}, we used our dataset to derive and explore 
the $M_{\rm BH} - L_{\rm gal}$ and $M_{\rm BH} - L_{\rm sph}$ (or $M_{\rm BH} - M_{\rm *,sph}$) diagrams. 
When considering all galaxies, irrespective of their morphological type, 
the $M_{\rm BH} - L_{\rm sph}$ correlation has a lower level of intrinsic scatter 
than the $M_{\rm BH} - L_{\rm gal}$ correlation. 
However, when considering only early-type galaxies (elliptical+lenticular), 
$L_{\rm gal}$ and $L_{\rm sph}$ correlate equally well with $M_{\rm BH}$. 
\citet{lasker2014anal} found the same level of intrinsic scatter in their 
$M_{\rm BH} - L_{\rm gal}$ and $M_{\rm BH} - L_{\rm sph}$ diagrams for all galaxies
because their sample was highly biased towards early-type objects 
(among their 35 galaxies, only four are spirals). 
Our subsample of S\'ersic spheroids defines a less steep $M_{\rm BH} - L_{\rm sph}$ 
sequence than previously reported \citep{grahamscott2013}. 
Looking at the spheroids with S\'ersic index $n_{\rm sph} < 2$, 
we did not observe any systematic offset 
from the $M_{\rm BH} - L_{\rm sph}$ correlation traced by spheroids with $n_{\rm sph} > 2$, 
in disagreement with the claims of \citet{sani2011}. 
We identified two distinct trends in the $M_{\rm BH} - M_{\rm *,sph}$ diagram, 
that is, a \emph{red sequence} of early-type (elliptical+lenticular) galaxies 
following $M_{\rm BH} \propto M_{\rm *,sph}^{1.04 \pm 0.10}$ 
and a dramatically steeper \emph{blue sequence} of late-type (spiral) galaxies 
following $M_{\rm BH} \propto M_{\rm *,sph}^{2-3}$. 
This is in agreement with the near-linear $M_{\rm BH} - L_{\rm sph}$ (or $M_{\rm BH} - M_{\rm *,sph}$) correlation 
measured by previous studies that used galaxy samples dominated by high-mass early-type objects 
(e.g.~\citealt{magorrian1998,marconihunt2003,haringrix2004,gultelkin2009,sani2011,beifiori2012,
erwingadotti2012,vika2012}),  
and it gives reason as to why other studies, 
whose galaxy samples gave a better representation of the late-type population 
with low-mass ($M_{\rm BH} \lesssim 10^8 \rm~M_\odot$) black holes, 
obtained a correlation steeper than linear 
or noticed super-linear deviations at the low-luminosity end of the diagram 
(e.g.~\citealt{laor1998,laor2001,wandel1999,salucci2000,ryan2007}). 
An interesting exception has been the under-linear $M_{\rm BH} \propto L_{\rm sph}^{0.75 \pm 0.10}$ relation 
measured by \citet{lasker2014anal}, 
who attributed the smaller log-slope of the correlation 
to the reduction of bulge luminosities derived from their multicomponents fits 
and to the type of linear regression algorithm used. 
It is worth mentioning that, among our 17 spiral galaxies, at least 12 host a Seyfert nucleus. 
\citet{wandel1999} and \citet{ryan2007} had previously noted that a linear $M_{\rm BH} - M_{\rm *,sph}$ relation 
systematically overestimates the $M_{\rm BH}/M_{\rm *,sph}$ ratio in Seyfert galaxies. 
In effect, the sample of $\approx 140$ low-redshift AGNs with virial black hole masses 
$10^5 \lesssim M_{\rm BH}/{\rm M_\odot} \lesssim 2 \times 10^6$ collected by \citet{grahamscott2015} 
appears to be the low-mass continuation of the $M_{\rm BH} - M_{\rm *,sph}$ sequence traced by our spiral galaxies. 
Finally, we note that our estimate of the normalization of the $M_{\rm BH} - M_{\rm *,sph}$ relation for early-type galaxies 
is the largest ever reported: $M_{\rm BH}/M_{\rm *,sph}$ is $\approx 0.7\%$ using our dataset, 
whereas it is as low as $\approx 0.5\%$ in \citet{scott2013} and \citet{kormendyho2013}, 
and even lower in previous studies. 
A recalibration of the $M_{\rm BH} - M_{\rm *,sph}$ normalization has important consequences 
for semi-analytical models and simulations of galaxy evolution that include black hole growth, 
and it also affects calculations of gravitational waves background from SMBH binary coalescence. 
However, it is important to keep in mind the warnings of \citeauthor{merritt2013book} 
(\citeyear{merritt2013book}; see also \citealt{ferraresemerritt2000,merrittferrarese2001ms,merrittferrarese2001sphinfl,
valluri2004,ferrareseford2005}), 
who cautioned that the majority of stellar-dynamics based black hole mass measurements 
are likely to have been overestimated by a factor of $3$ to $4$ 
due to an insufficient spatial resolution of the sphere-of-influence of black holes. 
His concern seems to be confirmed by the very recent findings of \citet{shankar2016}, 
who argued that the current sample of directly measured black hole masses 
is highly biased by the fact that their detectability 
is subject to the requirement of spatially resolve their gravitational sphere-of-influence. 
This selection effect allows us to see only the upper envelope of the ``true'' black hole mass correlations, 
artificially increasing the normalization of the $M_{\rm BH} - \sigma_*$ correlation by a factor of $\approx 3$ 
and even more dramatically that of the $M_{\rm BH} - M_{\rm *,sph}$ correlation. 
New black hole mass measurements obtained with the extraordinary spatial resolution of the next generation 
30-meter class telescopes will help shed light on this point.  \\

In Chapter \ref{ch:mn} \citep{paperIII}, 
we analysed the $L_{\rm sph} - n_{\rm sph}$ and $M_{\rm BH} - n_{\rm sph}$ diagrams 
and found that in the former early- and late-type galaxies split into two separate correlations, 
whereas in the latter relation all galaxies -- irrespective of their morphological type -- define a single correlation. 
Our $M_{\rm BH} - L_{\rm sph}$ and $M_{\rm BH} - n_{\rm sph}$ diagrams 
show consistent amount of intrinsic scatter, 
in agreement with the results from \citet{grahamdriver2007}. 
The present dataset allows one to explore also the correlation of black hole mass with 
spheroid central surface density, $\mu_{\rm 0, sph}$, 
and spheroid deprojected central stellar density, $\rho_{\rm *,0,sph}$. \\

In Chapter \ref{ch:msigma}, we focused on the $M_{\rm BH} - \sigma_*$ diagram 
and in particular on the over-massive black holes that have been reported to exist 
at the high-mass end of the correlation. 
\citet{volontericiotti2013} explained the presence of the outlying black holes 
as a natural consequence of the fact that their host Central Cluster Galaxies (CCGs) have experienced more dry mergers 
than any other galaxy, 
since dry mergers are expected to increase the black hole mass 
and ``conserve'' the stellar velocity dispersion. 
We tested and disproved this scenario for 23 core-S\'ersic galaxies, 
using the ratio between black hole mass and stellar mass deficit as a proxy 
for the number of dry mergers experienced by a galaxy \citep{merritt2006}. 
This was confirmed by a second test using the central kinematic classification of 37 galaxies. 
Our analysis suggests that 
the merger history of the CCGs hosting over-massive black holes in the $M_{\rm BH} - \sigma_*$ diagram 
is not exceptional, 
therefore these galaxies should be considered as legitimate members of the $M_{\rm BH} - \sigma_*$ correlation. 
An alternative possibility for the CCGs is that their partially depleted cores 
have been replenished by stars, globular clusters and nuclear star clusters 
acquired from minor dry mergers, 
where the accreted satellite galaxies did not host a massive black hole. 
However, high-resolution cosmological simulations seem to favour an ``inside-out'' formation scenario 
for early-type galaxies, 
where minor mergers only contribute to the build-up of the external envelope of high-mass galaxies 
(e.g.~\citealt{wellons2016}). 
It should also be noted that evidence is accumulating against the ``naive'' assumption 
that low-mass galaxies do not host any central massive black hole 
(e.g.~\citealt{Baldassare2015,Graham2016LEDA}). \\

Finally, in Chapter \ref{ch:ellic}, 
we tackled the issue of the over-massive black holes in the $M_{\rm BH} - M_{\rm *,sph}$ diagram. 
We showed that the intermediate-scale discs of the galaxies Mrk 1216 \citep{yildirim2015}, NGC 1271 \citep{walsh2015}, 
NGC 1277 \citep{vandenbosch2012}, and NGC 1332 \citep{rusli2011} 
had been confused with large-scale discs and incorrectly modelled as such. 
This led to a considerable underestimation of their spheroid luminosity, 
which caused them to appear as extreme outliers above the $M_{\rm BH} - M_{\rm *,sph}$ correlation 
of local early-type galaxies. 
Furthermore, at least one of these galaxies had its black hole mass overestimated by one order of magnitude 
(NGC 1277, e.g.~\citealt{Graham2016n1277}). 
\cite{gds2015} showed that, while the majority of the high-redshift compact massive spheroids 
have evolved into today's bulges of lenticular and early-type spiral galaxies (see also \citealt{graham2013review}), 
i.e.~they have accreted a large-scale stellar disc, 
a few of them (such as Mrk 1216, NGC 1271, NGC 1277, NGC 1332, and NGC 3115) 
have only accreted an intermediate-scale stellar disc. 
The spheroidal components of the galaxies Mrk 1216, NGC 1271, NGC 1277, NGC 1332, and NGC 3115 
have the same structural properties (compactness), stellar mass ($\approx 10^{11} \rm~M_\odot$), 
high stellar velocity dispersion ($\approx 300 \rm~km~s^{-1}$) and purely old stellar population 
as the $z \approx 2$ compact massive spheroids (e.g.~\citealt{daddi2005,trujillo2006,vandokkum2008,damjanov2009}), 
i.e.~they experienced negligible evolution over the last $10 \rm~Gyr$. 
If also their black holes have had little or no mass accretion since $z = 2$, 
assuming that the $M_{\rm BH} - M_{\rm *,sph}$ relation was already in place $10 \rm~Gyr$ ago, 
these five spheroids provides us with a fossil record of the correlation at $z=2$. 
The fact that the five spheroids lie within $\lesssim 1\sigma$ above the $z=0$ correlation 
can be used as a constraint for the evolution with redshift of the $M_{\rm BH} - M_{\rm *,sph}$ normalization 
(assuming no evolution with redshift of the log-slope of the correlation). 
Furthermore, knowing that their intermediate-scale discs have been accreted over the last $10 \rm~Gyr$ 
and that their disc-to-total ratios are typically small ($\lesssim 20\%$), 
one can deduce that the location of these five galaxies in the $M_{\rm BH} - M_{\rm *,gal}$ diagram 
(where $M_{\rm *,gal}$ is the galaxy stellar mass) 
is roughly the same as that at $z=2$. 
Therefore, finding a larger number of local galaxies with the same charachteristics as the aforementioned five 
would be helpful to study the redshift evolution of the $M_{\rm BH} - M_{\rm *,sph}$ 
and $M_{\rm BH} - M_{\rm *,gal}$ relations. 
The Hobby-Eberly Telescope Massive Galaxy Survey (HETMGS, \citealt{vandenBosch2015}) 
has identified a few of these rare objects 
such as Mrk 1216, NGC 1271, NGC 1277, and NGC 1281 \citep{yildirim2016}, 
although the last galaxy still lacks an accurate photometric decomposition; 
more of them might be yet to be found in the HETMGS database. 
\citet{saulder2015} selected 76 compact massive galaxies with high stellar velocity dispersion within $0.05 < z < 0.2$ 
from the Sloan Digital Sky Survey (SDSS, \citealt{abazajian2009}) 
and proposed them as candidate survivors of the $z=2$ compact massive spheroids. 
However, it should be noted that their criterion of high mass and compactness relies on the \emph{galaxy} properties 
rather than the \emph{spheroid} properties, 
therefore a number of these candidates might not be real descendants of the $z=2$ population\footnote{Some 
candidates might be lenticular galaxies with compact \emph{galaxy} sizes and large disc-to-total ratios, 
therefore their spheroids would be significantly less massive than the $z=2$ red nuggets. }. 
The same concern holds for the simulated compact massive galaxies of \citet{wellons2016}. 
It would be interesting to perform photometric decompositions 
for the observed 76 compact massive galaxies of \citet{saulder2015} 
and for those simulated by \citet{wellons2016}, 
to select only those with compact massive \emph{spheroids}. 
\citet{kormendy2011} and \citet{kormendyho2013} argued that black holes correlate only 
with the remnants of major mergers, i.e.~classical bulges and elliptical galaxies, 
and that the growth of the most massive black holes happened by means of a co-evolution with their host galaxies 
via a series of mergers and AGN feedback. 
The universality of this picture is contradicted by galaxies like Mrk 1216, NGC 1271, NGC 1277, NGC 1332, and NGC 3115. 
These galaxies have classical bulges and  
host some of the most massive black holes known ($10^9 \lesssim M_{\rm BH}/{\rm M_\odot} \lesssim 10^{10}$), 
but they are not the product of major mergers like the majority of today's massive elliptical galaxies. 
In effect, these galaxies have conserved their $10 \rm~Gyr$ old compact massive spheroids 
because they had little or no evolution since $z \approx 2$; 
note that $z \approx 3$ is the epoch that corresponds to the highest cosmic merger rate 
for galaxies more massive than $10^{10} \rm~M_\odot$ (e.g.~\citealt{conselice2008}). 


\vspace{5cm}
\begin{flushright}
\emph{Keep looking up. }
\end{flushright}

 
