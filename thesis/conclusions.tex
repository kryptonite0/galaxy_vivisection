\chapter{Final remarks}
\label{ch:concl}

In this thesis, we explored scaling relations between the supermassive black hole mass 
and various properties of the host spheroid, 
aiming at gaining a more profound understanding of the co-evolution between SMBHs and galaxies. 
A summary of our principal findings and some additional brief considerations about them are presented here. 
The end of this Chapter incorporates some promising future research directions. \\

In Chapter \ref{ch:recov-mn} \citep{savorgnan2013}, we compared the results obtained by four independent studies 
\citep{grahamdriver2007,sani2011,vika2012,beifiori2012}
that attempted photometric decompositions of similar samples of galaxies with a direct measurement 
of the black hole mass. 
In many cases we found a large discrepancy between the S\'ersic index measurements 
obtained by different studies for the spheroidal component of the same galaxy, 
either due to a significantly different choice of model components 
or to other various systematic effects. 
By rejecting the most discrepant S\'ersic index measurements and averaging the remaining ones, 
we were able to recover a strong $M_{\rm BH} - n_{\rm sph}$ correlation, 
which was not found using the individual datasets of three of the four studies. 
From this we concluded that some of the galaxy decompositions were not accurate. 
Chapter \ref{ch:recov-mn} emphasises the importance of a correct, physically motivated selection of model components 
and cautions against the several systematics that can affect galaxy decomposition. \\

Chapter \ref{ch:galviv} was dedicated to the careful multicomponent decomposition of 66 galaxies 
with a direct measurement of the black hole mass. 
We followed the same approach as \citet{laurikainen2005} in selecting the components for each galaxy model.  
\emph{A priori} identification of the galaxy components was done on the basis of several different indicators 
such as the analysis of the isophotal parameters, 
the inspection of unsharp masks, 
complementary information extracted from the literature, 
and -- most importantly -- from the galaxy kinematics. 
A joint photometric-kinematic approach (e.g.~\citealt{krajnovic2013,arnold2014}) 
turned out to play a decisive role for the robustness of our galaxy modelling. 
Upon examining central \citep{atlas3dIII,scott2014} and more extended \citep{arnold2014} velocity maps 
and comparing kinematic and photometric signatures of stellar discs, 
we were able to securely identify the presence and the radial extent of such discs. 
We observed a wide range of spheroid-to-disc ratios in early-type galaxies, 
going from parsec-sized nuclear discs, to kiloparsec-sized intermediate- and large-scale discs, 
in agreement with the findings of \citet{krajnovic2013}. 
Comparison of our results with those obtained by previous studies indicates that 
multicomponent models (as opposed to simple bulge/disc models) are necessary 
to derive reliable structural parameters, confirming previous results 
(e.g.~\citealt{laurikainen2005,laurikainen2007,laurikainen2010,lasker2014data,salo2015}). 
In general, the best-fit parameters obtained with 1D and 2D decomposition techniques for the same galaxy 
are consistent with each other, 
i.e.~no systematic effects were noticed between 1D and 2D modelling. 
However, our practical experience led us to prefer the 1D decomposition technique. 
Advantages associated with the 1D technique are a higher convergence rate for the fits, 
the wealth of information contained in the 1D isophotal analysis, 
and the easier interpretation of the 1D residuals. 
We caution against the dangerous practice of identifying unsubtracted galaxy components 
from the residual image of a 2D fit. 
As an additional warning, 
given the level of detail to which our galaxy decompositions were carried out, 
we do not consider possible for current automatic routines to reproduce our analysis. \\

In Chapter \ref{ch:mm}, we used our dataset to derive and explore 
the $M_{\rm BH} - L_{\rm gal}$ and $M_{\rm BH} - L_{\rm sph}$ (or $M_{\rm BH} - M_{\rm *,sph}$) relations. 
We identified two distinct trends in the $M_{\rm BH} - M_{\rm *,sph}$ diagram, 
that is, a \emph{red sequence} of early-type galaxies (E+S0) 
following $M_{\rm BH} \propto M_{\rm *,sph}^{1.04 \pm 0.10}$ 
and a dramatically steeper \emph{blue sequence} of late-type galaxies (Sp) 
following $M_{\rm BH} \propto M_{\rm *,sph}^{2-3}$.


 
